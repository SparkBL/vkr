\section {Вычисление асимптотических результатов}
В данной главе будет рассмотрен метод вычисления коэффициента вариации длин интервалов между моментами поступления заявок входящего потока системы массового обслуживания. Помимо этого, будет предложен метод вычисления вероятностей при помощи дискретного преобразования Фурье.
\subsection{Использование дискретного преобразования Фурье}
Ранее в работе \cite{blaginin2021approximation} при вычислении распределения вероятностей числа обслуженных заявок с помощью асимптотических формул использовалось обратное преобразование Фурье для дискретных случайных величин. Однако полученные формулы подходят лишь для точечных проверок из-за сложности вычисления. В случае с двумерным распределением вероятностей, процесс подсчета плоскость размером 10 на 10 точек может занимать около минуты. Чтобы решить проблему чрезвычайно долгого вычисления было принято решение использовать дискретное преобразование Фурье с заданной точностью \cite{nussbaumer1981fast,bergland1969guided}. Особенность данного метода заключается в выборе такого шага дискретизации $\delta$, чтобы в результате преобразования основная часть результирующего распределения не повторялась и совпадала с искомым. Шаг дискретизации был выбран согласно интервалу интегрирования и вычисляется следующим образом
\begin{equation*}
	\delta = \frac{2\pi}{n},
\end{equation*}
где $n$ --- длина результирующего вектора. Таким образом, для корректного преобразования требуется подобрать длину вектора таким образом, чтобы оно охватывало часть распределения в районе математического ожидания в диапазоне значения дисперсии. 

Были рассмотрены одномерный и двумерный случаи преобразования
\begin{equation*}
	B_n =\frac{1}{\sqrt{n}} \sum_{k=0}^{N-1} W_k \cdot e^{-2\pi\j (n/N)k}
\end{equation*}
\begin{equation*}
	B_{n,m} =\frac{1}{\sqrt{nm}} \sum_{k=0}^{N-1}\sum_{h=0}^{M-1} W_{k,h} \cdot e^{-2\pi\j (n/N)k -2\pi\j(m/M)h} 
\end{equation*}
Была написана реализация на языке Python для одномерного и двумерного случаев. В качестве вспомогательной библиотеки для операций над векторами использовалась numpy \cite{harris2020array}.
\lstset{language=Python,
	basicstyle=\linespread{0.8}\ttfamily,
	caption={Реализация ДПФ на Python}
}
\begin{lstlisting}
import numpy as np

def icfft(n,vector):
	d = np.zeros(n,dtype=complex)
	for j in range(0,n-1):
		for k in range(0,n-1):
			d[j] += vector[k]*np.exp(-2*np.pi*1j*(j/n)*k)
		d[j] *= 1/np.sqrt(n)
	return d

def icfft2(n,m,matrix):
	d = np.zeros((n,m),dtype=complex)
	for j in range(0,n-1):
		for i in range(0,m-1):
			for k in range(0,n-1):
				for h in range(0,m-1):
					d[j][i] += matrix[k][h]*np.exp(-2*np.pi*1j*(j/n)*k - 2*np.pi*1j*(i/m)*h)
			d[j][i] *= 1/np.sqrt(n*m)
	return d
\end{lstlisting}

\lstset{language=Python,
	basicstyle=\linespread{0.8}\ttfamily,
	caption={Вычисление ДПФ на Python}
}
\begin{lstlisting}
length =30
iii = list(range(0,length-1))
delta = (2*np.pi)/length
res = []
for i in iii:
res.append(complex(s.SFF(-np.pi + i * delta,10)))
ifft = [abs(n)/np.sqrt(length) for n in icfft(length,res)]
\end{lstlisting}
На рисунке \ref{ifft_plot_test} видно, что полученное преобразование (IFFT) полностью совпадает с результатом вычисления при помощи интегрирования (SDST)
\begin{figure}[H]
	\centering
	\includegraphics[scale=0.8,width=\textwidth]{ifft_plot_test.png}
	\caption{Сравнение распределений вероятностей, полученных с помощью интегрирования и ДПФ}
	\label{ifft_plot_test}
\end{figure}
Для проверки скорости работы данного подхода к вычислению был проведен ряд тестов (150 запусков для одномерного и двумрного случая) со сравнением скорости работы алгоритмов и точности получаемого распределения при помощи расстояния Колмогорова
\begin{equation*}
	\Delta = \underset{0 < i < \infty}{max}\bigg\rvert \sum_{v=0}^{i} (P_0(v) - P_1(v))\bigg\rvert.
\end{equation*}
Были получены следующие результаты:
\begin{itemize}
	\item Для одномерного случая в среднем ДПФ быстрее интегрирования в 411.717 раза, среднее расстояние Колмогорова --- 4.164439016729942e-06.
	\item Для двумерного случая в среднем ДПФ быстрее интегрирования в 901.184 раза, среднее расстояние Колмогорова --- 2.157044454159e-06.
\end{itemize}
Так, можно заключить, что дискретное преобразование Фурье не проигрывает в точности и в то же время существенно быстрее, что является важным результатом для последующей работы.
\subsection{Вычисление коэффициента вариации MMPP}
В данной работе одним из объектов изучения является вариация длин интервалов между моментами поступления заявок MMPP.

Согласно \cite{вишневский2018стохастические}, вариация длин интервалов между моментами поступления заявок MMPP вычисляется как
\begin{equation}
	Var = \frac{\sqrt{v}}{Lg^{-1}},
\end{equation}
где $v$ --- дисперсия длин интервалов между моментами поступления групп запросов и рассчитывается как
\begin{equation}
	v = \frac{2Lg\cdot r \cdot (-D0)^{-1} \cdot E -1}{Lg^2},
\end{equation}
где вектор--строка $r$ --- стационарное распределение вероятностей процесса\\$\{k(t),n(t)\}$, $E$ --- единичный вектор--столбец размерности N, $Lg$ -- интенсивность входящего потока
\begin{equation} \label{eq_lg}
	Lg = r\cdot \Lambda \cdot E,
\end{equation}
\begin{equation*}
	D0 = Q - \Lambda - Q\cdot D,
\end{equation*}
где $D$ --- матрица, содержащая вероятности наступления события в потоке при смене его состояния.
\clearpage
