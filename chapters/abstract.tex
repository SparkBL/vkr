\section*{\normalsize\centering АННОТАЦИЯ}
Работа содержит 68 страниц, 36 рисунков, 7 таблиц, 36 источников.
 
ТЕОРИЯ МАССОВОГО ОБСЛУЖИВАНИЯ, СИСТЕМА МАССОВОГО ОБСЛУЖИВАНИЯ С ПОВТОРНЫМИ ВЫЗОВАМИ И ВЫЗЫВАЕМЫМИ ЗАЯВКАМИ, МЕТОД АСИМПТОТИЧЕСКОГО АНАЛИЗА, ИМИТАЦИОННОЕ МОДЕЛИРОВАНИЕ.

Объект исследования --- двумерный выходящий поток модели узла обработки запросов с повторными обращениями, вызываемыми заявками и разными моделями входящего потока обращений.
Методы исследования --- метод асимптотического анализа, метод имитационного моделирования. 

Результаты работы --- для рассмотренных моделей систем массового обслуживания получены асимптотические приближения двумерной характеристической функции числа обслуженных заявок при условии большой задержки на орбите. На их основе получены формулы для расчета распределения вероятностей числа обслуженных заявок и коэффициента корреляции компонентов выходящего потока. Разработан и реализован программный продукт, позволяющий проводить имитационное моделирование рассмотренных моделей систем. Проведен численный анализ характеристик их работы.

Актуальность данной работы заключается в важности сведений о функционировании рассмотренных моделей систем в виде выходящего потока заявок. Системы с повторными вызовами с высокой точностью описывают функционирование ряда технологий в области телекоммуникационных сетей, а полученные результаты крайне важны при проектировании такого рода сетей и решении задач оптимизации.

В первом разделе рассматривается модель системы с простейшим входящим потоком, во втором разделе --- модель система с MMPP. В третьем разделе описана разработка, реализация и функционирование имитационной модели. В четвертом разделе проводится численный анализ асимптотических результатов с использование имитационной модели.

\thispagestyle{empty} % выключаем отображение номера для этой страницы
\clearpage