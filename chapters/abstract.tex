\section*{\normalsize\centering АННОТАЦИЯ}
Работа содержит \pageref{LastPage} страниц, \totalfigures\ рисунков, \totaltables\ таблицы, 67 источников.
 
ТЕОРИЯ МАССОВОГО ОБСЛУЖИВАНИЯ, СИСТЕМА МАССОВОГО ОБСЛУЖИВАНИЯ С ПОВТОРНЫМИ ВЫЗОВАМИ И ВЫЗЫВАЕМЫМИ ЗАЯВКАМИ, МЕТОД АСИМПТОТИЧЕСКОГО АНАЛИЗА, ИМИТАЦИОННОЕ МОДЕЛИРОВАНИЕ.

Объект исследования: оптимизация проведения численных экспериментов для систем теории массового обслуживания.
Методы исследования: метод имитационного моделирования, метод асимптотического анализа. 

Результаты работы: имитационное моделирование было рассмотрено как численный метод исследования систем массового обслуживания. Был спроектирован и реализован программный комплекс с набором инструментов для анализа систем теории массового обслуживания, содержащий имитационную модель, алгоритмы вычисления характеристик и вспомогательные утилиты. Был предложен метод обращения характеристических функций на основе дискретного преобразования Фурье для эффективного вычисления асимптотических результатов. На ряде примеров был проиллюстрирован подход к работе с программным комплексом и задачи, которые он позволяет решать.

Актуальность данной работы обусловлена универсальностью метода численного исследования систем, так как он может выступать и в качестве  отдельной методологии исследования, так и позволяет подтверждать имеющие аналитические результаты и делать косвенные выводы об аспектах работы системы, которые еще не были получены аналитически, но доступны к изучению численно. Для получения достоверных численных результатов в большинстве задач требуется проведение большого числа экспериментов при различных параметрах системы. Работа посвящена оптимизации процесса проведения численных экспериментов в условиях ограниченных вычислительных ресурсов и времени.

В первом разделе описан метод моделирования систем массового обслуживания и алгоритм для его проведения. Во второй главе описана архитектура, процесс разработки и особенности реализации программного комплекса. В третьей главе описаны подходы к обращению характеристических функций для работы с асимптотическими результатами исследования моделей. В четвертой главе описывается процесс работы с реализованным программным комплексом на примере исследования модели RQ---системы с повторными вызовами и обратной связью методами машинного обучения, где в качестве обучающей выборки выступают результаты имитационного моделирования.

\thispagestyle{empty}\addtocounter{page}{-1} % выключаем отображение номера для этой страницы
\clearpage