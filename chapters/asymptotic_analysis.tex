\section{Метод асимптотического анализа}
Полученную систему дифференциальных уравнений в частичных производных  (\ref{characteristic_equations}) будем решать методом асимптотического анализа в предельном условии большой задержки заявок на орбите ($\sigma \xrightarrow{} 0$).

Обозначим $\epsilon = \sigma,   u= \epsilon w,   F_{k}(w,u_{1},u_{2},t,\epsilon) = H_{k}(u,u_{1},u_{2},t)$, тогда система запишется в виде
\begin{equation} \label{asymptotic_equations}
	\begin{split}
		\frac{{\partial F_{0}(w,u_{1},u_{2},t,\epsilon)}}{{\partial t}} &= -(\lambda + \alpha)F_{0}(w,u_{1},u_{2},t,\epsilon) + j
		\frac{{\partial F_{0}(w,u_{1},u_{2},t,\epsilon)}}{{\partial w}} +\\  &+ \mu_{1} e^{ju_{1}}F_{1}(w,u_{1},u_{2},t,\epsilon) + \mu_{2}e^{ju_{2}}F_{2}(u,u_{1},u_{2},t,\epsilon) ,
		\\
		\frac{{\partial F_{1}(w,u_{1},u_{2},t,\epsilon)}}{{\partial t}} &= -(\lambda + \mu_{1})F_{1}(w,u_{1},u_{2},t,\epsilon) - j e^{-j\epsilon w}
		\frac{{\partial F_{0}(w,u_{1},u_{2},t,\epsilon)}}{{\partial w}} +\\  &+ \lambda F_{0}(w,u_{1},u_{2},t,\epsilon) + \lambda e^{j\epsilon w}F_{1}(w,u_{1},u_{2},t,\epsilon) ,
		\\
		\frac{{\partial F_{2}(w,u_{1},u_{2},t,\epsilon)}}{{\partial t}} &= -(\lambda + \mu_{2})F_{2}(w,u_{1},u_{2},t,\epsilon)  + \lambda e^{j\epsilon w}F_{2}(w,u_{1},u_{2},t,\epsilon) +\\  &+ \alpha F_{0}(w,u_{1},u_{2},t,\epsilon).
	\end{split}
\end{equation}  
Затем, что используя условие согласованности многомерных распределений, характеристическая функция процессов $m_{1}(t$) и $m_{1}(t)$ будет записана в следующем виде с введенными функциями 
\begin{equation*}
	M\{\exp(ju_{1}m_{1}(t))\exp(ju_{2}m_{2}(t))\}=\sum_{k=0}^{2}H_{k}(0,u_{1},u_{2},t) = \sum_{k=0}^{2}F_{k}(0,u_{1},u_{2},t,\epsilon).
\end{equation*}

\begin{theorem}
Асимптотические приближение двумерной характеристической функции числа обслуженных заявок входящего потока и числа обслуженных вызванных заявок за некоторое время $t$ имеет вид
\begin{equation*} \label{theor}
	\begin{split}
	\boldsymbol{F}(u_{1},u_{2},t) =  \lim_{\sigma \xrightarrow{} 0} M\{\exp(ju_{1}m_{1}(t))\exp(ju_{2}m_{2}(t))\} &= 
	\\
= \lim_{\epsilon \xrightarrow{} 0} \sum_{k=0}^{2}F_{k}(0,u_{1},u_{2},t,\epsilon) = \boldsymbol{R} \cdot \exp\{G(u_{1},u_{2})t\} \cdot \boldsymbol{E}
	\end{split}
\end{equation*}
где 
\begin{equation*}
	\boldsymbol{G}(u_{1},u_{2})=\begin{bmatrix}
		-(\lambda + \alpha + \kappa) & \mu_{1}e^{ju_{1}} &  \mu_{2}e^{ju_{2}}\\
		\kappa+\lambda & -\mu_{1} & 0\\
		\alpha & 	0 &	-\mu_{2}
	\end{bmatrix}^{T},
\end{equation*}
вектор-строка $\boldsymbol{R}=\{R_{0},R_{1},R_{2}\}$ - стационарное распределение вероятности состояния прибора
 \begin{equation*}
 	\boldsymbol{R}=\{\frac{\mu_{2}(\mu_{1} - \lambda)}{\mu_{1}(\mu_{2} - \alpha)},\frac{\lambda}{\mu_{1}},\frac{\alpha(\mu_{1} - \lambda)}{\mu_{1}(\mu_{2} + \alpha)}\},
 \end{equation*}
$\kappa$ - нормированное среднее число заявок на орбите
\begin{equation*}
\kappa = \frac{\lambda(\lambda \mu_{2} + \alpha \mu_{1})}{\mu_{2}(\mu_{1} - \lambda)},
\end{equation*}
а $\boldsymbol{E}$ - единичный вектор-столбец соответствующей размерности.
\end{theorem}
\begin{proof}
 Делая предельный переход $ \lim_{\epsilon \xrightarrow{} 0} F_{k}(w,u_{1},u_{2},t,\epsilon) = F_{k}(w,u_{1},u_{2},t)$  в полученной системе (\ref{asymptotic_equations}) , система уравнений будет записана в виде
	\begin{equation} \label{eps}
		\begin{split}
			\frac{{\partial F_{0}(w,u_{1},u_{2},t)}}{{\partial t}} &= -(\lambda + \alpha)F_{0}(w,u_{1},u_{2},t) + j
			\frac{{\partial F_{0}(w,u_{1},u_{2},t)}}{{\partial w}} +\\  &+ \mu_{1} e^{ju_{1}}F_{1}(w,u_{1},u_{2},t) + \mu_{2}e^{ju_{2}}F_{2}(w,u_{1},u_{2},t) ,
			\\
			\frac{{\partial F_{1}(w,u_{1},u_{2},t)}}{{\partial t}} &= -(\lambda + \mu_{1})F_{1}(w,u_{1},u_{2},t) - j 
			\frac{{\partial F_{0}(w,u_{1},u_{2},t)}}{{\partial w}} +\\  &+ \lambda F_{0}(w,u_{1},u_{2},t) + \lambda F_{1}(w,u_{1},u_{2},t) ,
			\\
			\frac{{\partial F_{2}(w,u_{1},u_{2},t)}}{{\partial t}} &= -(\lambda + \mu_{2})F_{2}(w,u_{1},u_{2},t)  + \lambda F_{2}(w,u_{1},u_{2},t) +\\  &+ \alpha F_{0}(w,u_{1},u_{2},t).
		\end{split}
	\end{equation}  
Решение системы (\ref{eps}) будет получено в следующей форме
\begin{equation} \label{sol}
F_{k}(w,u_{1},u_{2},t) = \Phi(w)F_{k}(u_{1},u_{2},t).
\end{equation}  
$\Phi(w)$ - асимптотическое приближение характеристической функции числа заявок на орбите при условии большой задержки на орбите.

Подставив (\ref{sol}) в систему (\ref{eps}) и разделив обе части уравнений на $\Phi(w)$, получим
\begin{equation} \label{preres}
	\begin{split}
		\frac{{\partial F_{0}(u_{1},u_{2},t)}}{{\partial t}} &= -(\lambda + \alpha)F_{0}(u_{1},u_{2},t) + j
		\frac{\Phi'(w) }{\Phi(w)}F_{0}(u_{1},u_{2},t) +\\  &+ \mu_{1} e^{ju_{1}}F_{1}(u_{1},u_{2},t) + \mu_{2}e^{ju_{2}}F_{2}(u_{1},u_{2},t) ,
		\\
		\frac{{\partial F_{1}(u_{1},u_{2},t)}}{{\partial t}} &= -(\lambda + \mu_{1})F_{1}(u_{1},u_{2},t) - j 
		\frac{\Phi'(w) }{\Phi(w)}F_{0}(u_{1},u_{2},t) +\\  &+ \lambda F_{0}(u_{1},u_{2},t) + \lambda F_{1}(u_{1},u_{2},t) ,
		\\
		\frac{{\partial F_{2}(u_{1},u_{2},t)}}{{\partial t}} &= -(\lambda + \mu_{2})F_{2}(u_{1},u_{2},t)  + \lambda F_{2}(u_{1},u_{2},t) +\\  &+ \alpha F_{0}(u_{1},u_{2},t).
	\end{split}
\end{equation}  
Заметим, что $w$ содержится только в отношении $\frac{\Phi'(w) }{\Phi(w)}$, а остальные слагаемые и левые части уравнений не зависят от $w$. Это означает, что  $\Phi(w)$ имеет вид экспоненты. Учитывая, что  $\Phi(w)$ имеет смысл асимптотического приближения характеристической функции числа заявок на орбите, мы можем конкретизировать вид данной функции
\begin{equation*}
\frac{\Phi'(w) }{\Phi(w)} = \frac{e^{j\kappa w}j\kappa}{e^{j\kappa w}},
\end{equation*} 
где $\kappa$ - нормированное среднее число заявок на орбите, которое было получено в \cite{nazarov2017asymptotic} и имеет вид 
\begin{equation*}
	\kappa = \frac{\lambda(\lambda \mu_{2} + \alpha \mu_{1})}{\mu_{2}(\mu_{1} - \lambda)}.
\end{equation*}

Исходя из этого, система (\ref{preres}) примет следующий вид
\begin{equation} \label{res}
	\begin{split}
		\frac{{\partial F_{0}(u_{1},u_{2},t)}}{{\partial t}} &= -(\lambda + \alpha+ \kappa)F_{0}(u_{1},u_{2},t) + \\  &+ \mu_{1} e^{ju_{1}}F_{1}(u_{1},u_{2},t) + \mu_{2}e^{ju_{2}}F_{2}(u_{1},u_{2},t) ,
		\\
		\frac{{\partial F_{1}(u_{1},u_{2},t)}}{{\partial t}} &= (\lambda + \kappa)F_{0}(u_{1},u_{2},t) -  
		\mu_{1}F_{1}(u_{1},u_{2},t) +\\  &+  0F_{2}(u_{1},u_{2},t) ,
		\\
		\frac{{\partial F_{2}(u_{1},u_{2},t)}}{{\partial t}} &= \alpha F_{0}(u_{1},u_{2},t)   +  0F_{1}(u_{1},u_{2},t) -\\  &- \mu_{2}F_{2}(u_{1},u_{2},t).
	\end{split}
\end{equation}  
Введем следующие обозначения
\begin{equation*}
	\boldsymbol{F}(u_{1},u_{2},t) = \{F_{0}(u_{1},u_{2},t),F_{1}(u_{1},u_{2},t),F_{1}(u_{1},u_{2},t)\}
\end{equation*}  
\begin{equation*}
	\boldsymbol{G}(u_{1},u_{2})=\begin{bmatrix}
		-(\lambda + \alpha + \kappa) & \mu_{1}e^{ju_{1}} &  \mu_{2}e^{ju_{2}}\\
		\kappa+\lambda & -\mu_{1} & 0\\
		\alpha & 	0 &	-\mu_{2}
	\end{bmatrix}^{T},
\end{equation*}
$\boldsymbol{G}(u_{1},u_{2})$ - транспонированная матрица коэффициентов системы (\ref{res}).
Тогда получим следующее матричное уравнение
\begin{equation*}
	\frac{{\partial \boldsymbol{F}(u_{1},u_{2},t)}}{{\partial t}} =\boldsymbol{F}(u_{1},u_{2},t)\boldsymbol{G}(u_{1},u_{2}),
\end{equation*}
общее решение которого имеет вид
\begin{equation} \label{diff}
	\boldsymbol{F}(u_{1},u_{2},t)=\boldsymbol{C}e^{\boldsymbol{G}(u_{1},u_{2})t}.
\end{equation}
Для того, чтобы получить единственное решение, которое соответствует поведению рассматриваемой системы, примем в рассмотрение начальное условие
\begin{equation} \label{cond}
	\boldsymbol{F}(u_{1},u_{2},0)=\boldsymbol{R},
\end{equation}
где вектор-строка $\boldsymbol{R}$ - стационарное распределение вероятности состояния прибора, то есть процесса $k(t)$, которое имеет форму \cite{nazarov2017asymptotic}
\begin{equation*}
	\boldsymbol{R}=\{\frac{\mu_{2}(\mu_{1} - \lambda)}{\mu_{1}(\mu_{2} - \alpha)},\frac{\lambda}{\mu_{1}},\frac{\alpha(\mu_{1} - \lambda)}{\mu_{1}(\mu_{2} + \alpha)}\}.
\end{equation*}
Описав начальное условие, мы можем перейти к решению задачи Коши (\ref{diff}, \ref{cond}).

Поскольку нас интересует распределение вероятностей количества заявок в выходных процессах, необходимо найти маргинальное распределение. Для этого суммируем компоненты вектор-строки $\boldsymbol{F}(u_{1},u_{2},t)$ по $k$ и умножаем результат на единичный вектор-столбец $\boldsymbol{E}$. Получим
\begin{equation}\label{approx}
	\boldsymbol{F}(u_{1},u_{2},t)\boldsymbol{E}=\boldsymbol{R}e^{\boldsymbol{G}(u_{1},u_{2})t}\boldsymbol{E}.
\end{equation}
Эта формула позволяет найти асимптотическое приближение характеристической функции количества вызванных и входящих заявок, обслуженных системой к некоторому моменту времени $t$. Другими словами, формула (\ref{approx}) является решением рассматриваемой системы. 
\end{proof}