\section*{\centering\normalsize ЗАКЛЮЧЕНИЕ}
\addcontentsline{toc}{section}{Заключение}

В рамках данной работы был спроектирован и разработан программный комплекс с набором инструментов для анализа систем теории массового обслуживания, содержащий имитационную модель, алгоритмы вычисления характеристик и вспомогательные утилиты. Согласно указанной цели работы был выполнен ряд задач.

Было рассмотрено имитационное моделирование как численный метод исследования систем массового обслуживания и задачи, которые оно позволяет решать. В частности, было отмечено, что численные методы позволяют находить взаимосвязь между наблюдаемые характеристиками работы системы и теми, которые еще не были получены или не могут быть получены аналитическим путем. Также был рассмотрен алгоритм моделирования, основывающийся дискретно--событийном подходе, где в процессе работы модели содержится очередь событий, которые последовательно наступают и меняют состояние системы. 

Была переработана имитационная модель с ориентацией на масштабные запуски и параллелизм. Была улучшена ее производительность и отказоустойчивость при задании параметров, затрудняющих процесс моделирования. Также была реализована возможность задания параметров для всех элементов системы из файла, который заранее содержит информацию о требуемых результатах.

В процессе разработки программного комплекса была выделена предметная область, построена ее объектная модель (рисунок  \ref{d_d}), описывающая структуру и взаимодействие ее сущностей, среди которых элемент модели (рисунок \ref{d_producer}), маршрутизатор (рисунок \ref{d_router}), генератор задержки, заявка и сборщик статистики. Введена система слотов (рисунок \ref{d_slot}), которая в совокупности с остальными сущностями предметной области позволяет создавать широкий набор различных моделей система массового обслуживания для моделирования и анализа численными методами. Ключевой особенностью предметной области является общий интерфейс для всех элементов модели, генераторов задержки и сборщиков статистики, что позволяет легко расширять набор реализаций каждого интерфейса без изменений уже имеющихся реализаций.

Для обеспечения производительности имитационной модели, в качестве языка программирования для реализации предметной области был выбран язык C++, что позволило оптимизировать работу программы для проведения масштабных параллельных вычислений. Для обеспечения интеграции имитационного моделирования в процесс анализа, результат разработки является расширительным пакетом для Python, что позволило совместить эффективность системного языка программирования и простоту скриптового языка. Таким образом, мы получили возможность конфигурировать и запускать множество моделей параллельно при помощи простого программного интерфейса, а дальнейшая работа с результатами моделирования может происходить непосредственно в среде Python при помощи сторонних инструментов, что значительно сокращает трудоемкость проведения численного анализа и время, требуемое для исследования.

Для работы с результатами моделирования были разработаны программные инструменты на языке Python, позволяющие генерировать необходимое количество наборов параметров, проводить множественные запуски имитационной модели и агрегировать полученные данные.

Помимо этого, в рамках программного комплекса были реализованы алгоритмы вычисления характеристик рассматриваемой модели системы массового обслуживания, включающие также вспомогательные функции для вычисления точности распределения вероятностей при помощи расстояния Колмогорова.Особенностью данного инструмента является наличие функций для подсчета распределения вероятностей при помощи дискретного преобразования Фурье. Их реализация позволяет существенно оптимизировать процесс проведения экспериментов, так как время работы алгоритмов во много раз превосходит ранее использованный подход.

На ряде примеров был описан процесс работы с программным пакетом, в частности было показано получение распределения вероятностей числа обслуженных заявок для RQ--системы с повторными вызовами и обратной связью (рисунок \ref{rq_system}) и его характеристик. Также на примере исследования зависимости числа обслуженных заявок и времени ожидания заявки до получения обслуживания в той же модели системы был описан подход к проведению масштабных параллельных запусков моделирования с различной конфигурацией и последующий анализ полученных данных.

В конечном итоге, разработанные инструменты позволили провести апробацию методов машинного обучения для расчета одной из характеристик работы системы --- вариации длин интервалов между моментами покидания заявками прибора. На основе выборки, составленной из 286 тысяч запусков имитационной модели, были обучены алгоритмы LinearRegression, RandomForest, GradientBoost и CatBoost для решения задачи регрессии. Были выявлены зависимости целевой переменной от ряда признаков, в частности, задержки заявок на орбите и коэффициента вариации входящего потока. Обученные модели позволяют получать результаты с достоверностью около 80\%. В дальнейшем исследовании стоит цель увеличить точность предсказаний за счет изменения процедуры генерации наборов параметров таким образом, чтобы распределение каждого признака было более равномерным и охватывало как можно большее количество возможных значений во избежание искажения результатов.

Часть результатов проведенного исследования были представлены в качестве докладов на двух международных конференциях:
\begin{itemize}
	\item XXIV International Conference on Distributed Computer and Communication Networks (DCCN) (сентябрь 2021, Томск) \cite{blaginin2021approximation};
	\item XX Международная конференции по информационным технологиям и математическому моделированию имени А.Ф. Терпугова (ITMM) (декабрь 2021, Томск);
\end{itemize}

 \clearpage