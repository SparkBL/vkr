\section*{\centering\normalsize ЗАКЛЮЧЕНИЕ}
\addcontentsline{toc}{section}{Заключение}
В рамках данной работы был рассмотрен ряд Марковских систем массового обслуживания с повторными вызовами и вызываемыми заявками, имеющих в качестве источника заявок простейший и MMPP--потоки. Согласно указанной цели исследования был выполнен ряд задач.

Были построены математические модели функционирования рассматриваемых узлов обработки запросов. В зависимости от типа системы, описывающий ее Марковский процесс имеет разную размерность. Так, для системы с одномерным выходящим потоком и простейшим входящим он будет трехмерным --- $\{k(t),i(t),m(t)\}$, при построении модели с разными двумя типами заявок для второго типа был добавлен процесс $m_{2}t$, а для системы с входящим потоком MMPP--потоком был добавлен процесс $n(t)$, описывающий состояние управляющей цепи MMPP. На основе Марковских процессов были составлены системы уравнений Колмогорова. Для нахождения характеристик выходящего потока был осуществлен переход к характеристическим функциям и применен метод асимптотического анализа при в предельном условии большой задержки заявок на орбите. В результате были получены формулы (\ref{approximation_summary},\ref{approximation_twodim},\ref{approximation_twodim_map}) для вычисления асимптотического приближения характеристической функции числа заявок, окончивших обслуживание в систему к моменту времени $t$.

Для вычисления значений распределения вероятностей числа обслуженных заявок были получены формулы (\ref{distr_simple_summary},\ref{distr_simple_twodim},\ref{distr_map_twodim}), в которых используется преобразование подобия матриц для вычисления матричной экспоненты и обратное преобразование Фурье, позволяющее перейти от характеристической функции к явному виду распределения вероятностей. Помимо этого, в разделе \ref{corr_section} были приведены выкладки для вычисления коэффициента корреляции компонентов двумерного распределения при помощи полученных асимптотических приближений характеристической функции числа обслуженных заявок. Применение указанных формул позволило проводить анализ решений в системе компьютерной алгебры Mathcad и при заданных параметрах системы получать численные характеристики работы системы.

Для оценки применимости полученных асимптотических результатов был разработан и реализован программный продукт, позволяющий проводить имитационное моделирование рассматриваемых систем. В первую очередь, были выделены сущности, принадлежащие к предметной области работы. Далее, была построена объектная модель программы, позволяющая расширять набор используемых элементов теории массового обслуживания при помощи общего для них интерфейса, формализующего изменение состояния элемента при наступлении очередного события в модели. Процесс моделирования, заключающийся в регистрации моментов наступления событий, позволил проводить моделирование итеративно и регистрировать необходимые для анализа характеристики работы системы. Реализация программы представляет собой две составных части - библиотеку, реализованная с учетом построенной объектной модели и оболочку для ее использования с графическим пользовательским интерфейсом. Раздельная реализация дает возможность легко добавлять новый функционал к имеющемуся.  

Процесс работы с программой предусматривает два подхода. В первом случае, по умолчанию, моделирование производится в реальном времени с настраиваемым интервалом таймера, который отсчитывает проведение следующей итерации моделирования. Данный подход дает возможность пользователю получать результаты моделирования и проводить их предварительный анализ с помощью встроенных графических средств, таких как трехмерное представление распределения вероятностей с возможностью масштабирования и выделения необходимой области с отсечением. Второй способ позволяет быстро получить результаты моделирования для последующего анализа с помощью других программных средств, для чего реализована функция экспорта данных распределения вероятностей в текстовый формат. 

Имитационная модель была протестирована на стабильность получаемых результатов при помощи критерия согласия Колмогорова. В ходе эксперимента значение расстояния Колмогорова не превышало 0.002, что говорит о высокой стабильности моделирования.

Из-за асимптотической природы полученных результатов встает вопрос об их применимости на практике в реальных системах, соответствующих рассматриваемых моделям. Для ее оценки была применена указанная имитационная модель и расстояние Колмогорова, показывающие, насколько соответствует эмпирическое распределение вероятностей предложенной модели. Расчеты показали, что при увеличении задержки заявок на орбите аналитические формулы дают более точное распределение вероятностей числа обслуженных заявок. Такой результат является закономерным ввиду того, что аналитические решения были получены при соответствующем асимптотическом условии, однако, даже при меньшей задержке заявок на орбите расчеты оказываются достаточно точными, чтобы их можно было применять на практике. Внимания требует так же распределение вероятностей, получаемое при большей загруженности системы. В таких условиях расчеты становятся еще более точными. В ходе проведенных численных экспериментов значение расстояния Колмогорова не превышало 0.066 --- данный результат был получен при увеличенной загрузке системы с MMPP--потоком при интенсивности возврата заявок с орбиты равной 10.

Для рассматриваемых систем был проведен численный анализ корреляции выходящих процессов, в ходе которого были выявлены некоторые закономерности изменения корреляции выходящих процессов при различных параметрах системы. В частности, при варьировании параметров для разных типов заявок коэффициент корреляции всегда меньше нуля и тем меньше, чем больше интенсивность поступления заявок, входящих либо вызываемых, в систему; в то время как при изменении параметров системы для одного типа заявок коэффициент корреляции принимает как положительные, так и отрицательные значения. Также были найдены параметры системы, при которых компоненты выходящего потока становятся независимыми. Эти результаты крайне важны для понимая функционирования выходящих процессов системы, однако для интерпретации этого открытия требуется дальнейшее исследование.

В конечном итоге, проведенное исследование позволяет применять полученные в ходе него аналитические решения и результаты численного анализа  для изучения функционирования и отладки реальных систем массового обслуживания \cite{deering1991icmp,nutt1982performance} с высокой точностью. Крайне важным и трудно исследуемым аспектом функционирования таких систем является выходящих поток обслуженных требований, поскольку он представляет собой характеристику работы системы в совокупности, а она, в свою очередь, зависит от других процессов, происходящих в рамках системы. Помимо этого, исследование выходящего потока требований и его характеристик является необходимым при построении телекоммуникационных сетей, поскольку обслуженные требования одного узла является входящими для другого, что для сетей значительного размера представляет собой еще более важный аспект проектирования.

Результаты проведенного исследования были представлены в качестве докладов и обсуждались в ряде конференций:
\begin{itemize}
\item XIX Международная конференции по информационным технологиям и математическому моделированию имени А.Ф. Терпугова (ITMM) (май 2020, Томск);
\item VIII Международная молодежная научная конференция «Математическое и программное обеспечение информационных, технических и экономических систем» (май 2021, Томск);
\item VII Международная конференция «Математика, её приложения и математическое образование» (МПМО) (сентябрь 2020, Улан-Удэ);
\item V Международная конференции по стохастическим методам (ISCM) (ноябрь 2020, Москва);
\item VII Международная молодеждная научная конференция \textquote{Математическое и программное обеспечение информационных, технических и экономических систем} (май 2020, Томск).
\end{itemize}
Также, на тему данного исследования опубликована статья \cite{blaginin2020two}.
 \clearpage