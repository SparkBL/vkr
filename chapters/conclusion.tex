\section*{\centering\normalsize ЗАКЛЮЧЕНИЕ}
\addcontentsline{toc}{section}{Заключение}

В рамках данной работы был спроектирован и разработан программный комплекс с набором инструментов для анализа систем теории массового обслуживания, содержащий имитационную модель, алгоритмы вычисления характеристик и вспомогательные утилиты. Согласно указанной цели работы был выполнен ряд задач.

Имитационное моделирование было рассмотрено как численный метод исследования систем массового обслуживания, и было описано, какие задачи оно позволяет решать. В частности, было отмечено, что численные методы позволяют находить взаимосвязь между наблюдаемыми характеристиками работы системы и теми, которые еще не были получены или не могут быть получены аналитическим путем. Также был рассмотрен алгоритм моделирования, основывающийся на дискретно--событийном подходе, где в процессе работы модели содержится очередь событий, которые последовательно наступают и меняют состояние системы. 

В процессе разработки программного комплекса была определена его архитектура, построена объектная модель предметной области (рисунок  \ref{d_d}), описывающая структуру и взаимодействие ее сущностей, среди которых: элемент модели (рисунок \ref{d_producer}), маршрутизатор (рисунок \ref{d_router}), генератор задержки, заявка и сборщик статистики. Введена система слотов (рисунок \ref{d_slot}), которая в совокупности с остальными сущностями позволяет создавать широкий набор различных систем массового обслуживания для моделирования и анализа. Ключевой особенностью выбранной архитектуры является общий интерфейс для всех элементов модели, генераторов задержки и сборщиков статистики, что позволяет легко расширять набор реализаций каждого интерфейса без изменений уже имеющихся реализаций.

Для обеспечения производительности имитационной модели, в качестве языка программирования для реализации был выбран C++, что позволило оптимизировать работу программы для проведения масштабных параллельных вычислений. Для обеспечения интеграции имитационного моделирования в процесс анализа результатом разработки является расширительным пакетом для Python, что позволило совместить эффективность системного языка программирования и простоту скриптового языка. Таким образом, мы получили возможность конфигурировать и запускать множество моделей параллельно при помощи простого программного интерфейса, а дальнейшая работа с результатами моделирования может происходить непосредственно в среде Python при помощи сторонних инструментов, что значительно сокращает трудоемкость проведения численного анализа и время, требуемое для исследования. Помимо этого, инструмент содержит встроенную документацию или пояснение по работе с каждым классом или функцией, а установка возможна при помощи пакетного менеджера pip, что значительно упрощает начало работы с пакетом.

Помимо этого, в рамках программного комплекса были реализованы алгоритмы вычисления характеристик рассматриваемой модели системы массового обслуживания, включающие также вспомогательные функции для вычисления точности распределения вероятностей при помощи расстояния Колмогорова. Особенностью данного инструмента является наличие функций для подсчета распределения вероятностей при помощи дискретного преобразования Фурье. Их реализация позволяет существенно оптимизировать процесс проведения экспериментов.

На ряде примеров был описан процесс работы с программным пакетом, в частности было показано получение распределения вероятностей числа обслуженных заявок для RQ--системы с повторными вызовами и обратной связью (рисунок \ref{rq_system}) и его характеристик, а также проведение сравнения с результатов с асимптотическими. Также на примере исследования зависимости числа обслуженных заявок и времени ожидания заявки до получения обслуживания для той же модели системы был описан подход к проведению параллельных запусков моделирования с различной конфигурацией и последующий анализ полученных данных. Было описано, как можно генерировать необходимое количество наборов параметров, структурировать получаемую информацию и проводить ее анализ.

В конечном итоге, разработанные инструменты позволили провести апробацию методов машинного обучения для расчета одной из характеристик работы системы --- вариации длин интервалов между моментами покидания заявками прибора. Было проиллюстрировано, что оптимизация процесса проведения численных экспериментов позволила использовать новые методы исследования систем массового обслуживания --- на основе выборки, составленной из 286 тысяч запусков имитационной модели с различными параметрами, были обучены алгоритмы LinearRegression, RandomForest, GradientBoost и CatBoost для решения задачи регрессии. Были выявлены зависимости целевой переменной от ряда признаков, в частности, задержки заявок на орбите и коэффициента вариации входящего потока. Обученные модели позволяют получать результаты с достоверностью около 80\%. В дальнейшем исследовании стоит цель увеличить точность предсказаний за счет изменения процедуры генерации наборов параметров таким образом, чтобы распределение каждого признака было более равномерным и охватывало как можно большее количество возможных значений во избежание искажения результатов.

Часть результатов проведенного исследования была представлены в качестве докладов на двух международных конференциях:
\begin{itemize}
	\item XXIV International Conference on Distributed Computer and Communication Networks (DCCN) (сентябрь 2021, Томск) \cite{blaginin2021approximation};
	\item XX Международная конференции по информационным технологиям и математическому моделированию имени А.Ф. Терпугова (ITMM) (декабрь 2021, Томск);
\end{itemize}

 \clearpage