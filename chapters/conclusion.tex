\section*{\centering\normalsize ЗАКЛЮЧЕНИЕ}
\addcontentsline{toc}{section}{Заключение}
В рамках данной работы был рассмотрен ряд методов оптимизации экспериментального исследования модели системы массового обслуживания с повторными вызовами и вызываемыми заявками. Согласно указанной цели исследования был выполнен ряд задач.

Была переработана имитационная модель с ориентацией на масштабные запуски и параллелизм. Была улучшена ее производительность и отказоустойчивость при задании параметров, затрудняющих процесс моделирования. Также была реализована возможность задания параметров для всех элементов системы из файла, который заранее содержит информацию о требуемых результатах.

Для работы с результатами моделирования были разработаны программные инструменты на языке Python, позволяющие генерировать необходимое количество наборов параметров, проводить множественные запуски имитационной модели и агрегировать полученные данные.

Помимо этого, в рамках программного комплекса были реализованы алгоритмы вычисления характеристик рассматриваемой модели системы массового обслуживания, включающие также вспомогательные функции для вычисления точности распределения вероятностей при помощи расстояния Колмогорова. Ключевой особенностью данного инструмента является наличие функций для подсчета распределения вероятностей при помощи дискретного преобразования Фурье. Их реализация позволяет существенно оптимизировать процесс проведения экспериментов, так как время работы алгоритмов во много раз превосходит ранее использованный подход.

В конечном итоге, разработанные инструменты позволили провести апробацию методов машинного обучения для расчета одной из характеристик работы системы --- вариации длин интервалов между моментами покидания заявками прибора. На основе выборки, составленной из 286 тысяч запусков имитационной модели, были обучены алгоритмы LinearRegression, RandomForest, GradientBoost и CatBoost для решения задачи регрессии. Были выявлены зависимости целевой переменной от ряда признаков, в частности, задержки заявок на орбите и коэффициента вариации входящего потока. Обученные модели позволяют получать результаты с достоверностью около 80\%. В дальнейшем исследовании стоит цель увеличить точность предсказаний за счет изменения процедуры генерации наборов параметров таким образом, чтобы распределение каждого признака было более равномерным и охватывало как можно большее количество возможных значений во избежание искажения результатов.

Часть результатов проведенного исследования были представлены в качестве докладов на двух международных конференциях:
\begin{itemize}
	\item XXIV International Conference on Distributed Computer and Communication Networks (DCCN) (сентябрь 2021, Томск) \cite{blaginin2021approximation};
	\item XX Международная конференции по информационным технологиям и математическому моделированию имени А.Ф. Терпугова (ITMM) (декабрь 2021, Томск);
\end{itemize}

 \clearpage