\section*{ВВЕДЕНИЕ}
\addcontentsline{toc}{section}{Введение}
В современных условиях развития информационных технологий в обществе значительно выросла роль как Интернета, так и в общем случае получения доступа к различного рода распределенным ресурсам. В качестве таких ресурсов могут выступать, например, электронные записи к врачу, которые должны предоставлять людям множественный доступ к публичной информации, онлайн--консультации, обзору предстоящих событий, службе поддержки. Помимо проникших во всех сферы жизни общества информационных сервисов, надобность в распределенных ресурсах возникает в специализированных областях жизнедеятельности и науки --- облачные вычисления, автоматизированное производство, сетевые протоколы передачи информации и др.

В таком случае, для достижения эффективности доступа к ресурсам требуется решить ряд задач, связанных со спецификой самого ресурса, например, какое время требуется для обработки запроса, в какое время ресурс будет доступен, как сократить количество утерянных или неучтенных запросов. Ко всему неизвестно, как и когда именно запросы будут приходить. Именно таким классом задач занимается теория массового обслуживания --- изучение и моделирование при помощи математического аппарата различных ситуаций и схем доступа к распределенных ресурсам \cite{nazarov2010theory}. Теория массового обслуживания появилась благодаря  А.К. Эрлангу, занимавшемуся задачами оптимизации линий телефонной связи \cite{erlang1909theory}.

Теория массового обслуживания оперирует такими понятиями как заявка, буфер, прибор, орбита и другие. С их помощью описывается модель системы, как раз состоящая, в общем случае, из входящего потока заявок и обслуживающего прибора. Существует множество разновидностей моделей систем \cite{phung2019retrial,artalejo2010accessible}, описывающих, соответственно различные ситуации обслуживания или доступа к ресурсам в реальных жизненных ситуациях. В частности такие модели применяются для моделирования работы сетевых протоколов модели OSI (ICMP, CSMA, AMQP, Ethernet и др.) \cite{bellovin2003icmp,bjornstad2006traffic,kritzinger1986performance,olypher2010computer} и используются для их оптимизации и устранения заторов заявок при их резком увеличении в течение краткого временного интервала. Для подобной ситуации в теории массового обслуживания используют входящий поток заявок с управляющей марковской цепью --- MMPP \cite{baiocchi1993steady,2019asymptotic}. Она модулирует интенсивность поступления заявок во времени, тем самым имитируя свойственную реальному миру неоднородность поступления заявок при работе в сети.
 
Наряду с математическим аппаратом, имитационное моделирование \cite{задорожный2011методы} является эффективным методом исследования как моделей систем теории массового обслуживания, так и в широком смысле технологией системного анализа. Само имитационное моделирование возникло ввиду двух факторов: развитие вычислительной мощности ЭВМ, давшее возможность эффективно использовать численные метода анализа, и недостаточность математического аппарата для исследования систем, выражающейся в необходимости экспериментального подтверждения получаемых аналитических результатов, ввиду того, что они были получены при определенном ограничивающем условии \cite{горбунов2007парадигмы}. Однако, имитационное моделирование базируется на математической конструкции и опирается на ее логику, но при решении практической задачи применяются более гибкие методы, приводящие непосредственно к результатам численного эксперимента, в частности это касается выбора алгоритмов и различных архитектурных решений в модели.

Существует несколько методологий моделирования, но основными являются дискретно-событийное моделирование \cite{илюхина2015дискретно,григорьева2014дискретно} и агентное моделирование \cite{лебедюк2017агентное}. Дискретно-событийное моделирование заключается в последовательной обработке событий, происходящих в системе, в хронологическом порядке. Из специфики и показателей обработки событий и складывается общая характеристика работы системы. Агентное моделирование, в свою очередь, представляет собой взаимодействие отдельных элементов и подсистем, называемых агентами, которые снабжены своей собственной логикой функционирования и исполняют ее асинхронно и самостоятельно. Для экспериментального исследования моделей систем теории массового обслуживания, как правило, применяется дискретно событийный подход, так как заявки могут быть лаконично представлены в виде событий, происходящих в системе.

 Численные методы позволяют в полной мере исследовать рассматриваемую модель, а именно ответить на вопросы, касающиеся определения границ области применимости и наблюдения функционирования системы при нестандартных параметрах. Однако для получения достоверных результатов требуется проведение большого числа численных экспериментов и наличие специализированных инструментов для их анализа. Зачастую это является основных ограничением в ходе имитационного моделирования --- имея ограниченный объем вычислительных ресурсов и времени необходимо провести внушительное количество численных экспериментов. Поэтому данная работа посвящена оптимизации процесса экспериментального исследования моделей систем массового обслуживания различного рода. 
 
 Целью работы является разработка и реализация программного комплекса с набором инструментов для анализа систем теории массового обслуживания, который будет включать в себя алгоритмы вычисления характеристик, имитационную модель и средства для её эффективного использования.

Решение разработать программное обеспечение для моделирования было принято ввиду ряда аспектов исследовательской работы, которые требуют создания и анализа большего объема данных. Для этого требуется автоматизировать существенную часть процесса исследования ввиду ограниченного времени и ресурсов. Помимо этого, трудоемкость на последующих этапах работы значительно снижается благодаря структурированности получаемой из модели информации.

В рамках цели данной работы поставлены следующие задачи:
\begin{enumerate}
	\item Выработать функциональные требования, учитывающие быстродействие, параллельность вычислений и возможность точечной конфигурации каждой модели, к программному комплексу;
	\item Разработать архитектуру имитационной модели и логику ее функционирования;
	\item Реализовать модель с учетом разработанной архитектуры и требований;
	\item Реализовать метод ускоренного обращения характеристической функции на основе ее дискретизации;
	\item Разработать и реализовать вспомогательные программные инструменты для проведения численных экспериментов и статистической обработки их результатов;
	\item Провести численные эксперименты и предоставить результаты, иллюстрирующие возможности разработанного программного комплекса.
\end{enumerate}

Работа содержит \pageref{LastPage} страниц, \totalfigures\ рисунков, \totaltables\ таблицы, 93 источника.

В первом разделе описано описан метод моделирования систем массового обслуживания и алгоритм для его проведения. Во второй главе описана предметная область, процесс разработки и особенности реализации программного комплекса. В третьей главе описаны подходы к обращению характеристических функций для работы с асимптотическими результатами исследования моделей систем. В четвертой главе описывается процесс работы с реализованным программным комплексом на примере исследования модели RQ---системы с повторными вызовами и обратной связью и использования машинного обучения, где в качестве обучающей выборки выступают результаты имитационного моделирования.
 \clearpage
 