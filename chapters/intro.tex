\section*{ВВЕДЕНИЕ}
\addcontentsline{toc}{section}{Введение}

Данная работа посвящена оптимизации процесса экспериментального исследования модели системы массового обслуживания с повторными вызовами и вызываемыми заявками \cite{phung2019retrial} и MMPP \cite{baiocchi1993steady,2019asymptotic,meier1987fitting}.


Почему это надо

Обзора литературы

Что это позволит делать

На чем сделан упор в работе

Цель

Требования

Задачи

Тезисы введения:
\begin{enumerate}
	\item Используя наработки по ранее разработанной имитационной модели, оптимизировать ее для масштабного проведения численных экспериментов.
	\item Реализовать метод вычисления асимптотического распределения вероятностей, использую дискретное преобразование Фурье. 
	\item Реализовать алгоритм вычисления вариации для интервалов между моментами поступлениями заявок MMPP.
	\item Разработать и реализовать вспомогательные программные инструменты для проведения численных экспериментов.
	\item Провести апробацию машинного обучения для вычисления вариации длин интервалов между моментами поступления заявок из выходящего процесса рассматриваемой модели системы.
\end{enumerate}



Данная работа посвящена оптимизации процесса экспериментального исследования модели системы массового обслуживания с повторными вызовами и вызываемыми заявками \cite{phung2019retrial} и MMPP \cite{baiocchi1993steady,2019asymptotic,meier1987fitting}. Ранее для данной модели были получены асимптотические формулы для вычисления различных характеристик \cite{blaginin2020two,blaginin2021approximation}, однако процесс вычисления оказался гораздо более трудоёмким, чем было бы допустимо при наличии ограниченного времени и вычислительных ресурсов. Чтобы в полной мере исследовать рассматриваемую модель, а именно ответить на вопросы, касающиеся определения границ области применимости и наблюдения функционирования системы при нестандартных параметрах, требуется проведение большого числа численных экспериментов и наличие специализированных инструментов для их анализа. По этой причине было принято решение реализовать программный комплекс с набором инструментов для анализа рассматриваемой модели системы, который будет включать в себя алгоритмы вычисления характеристик, имитационную модель и средства для её эффективного использования.

Ранее при исследовании предложенной модели использовался подход, основанный на точечных экспериментах, однако он не позволил получать в достаточной степени достоверные результаты ввиду специфики рабочего процесса с имитационной моделью. Поэтому подход к проведению экспериментов был изменен в сторону более масштабных вычислений, чтобы позволить проводить их массово с эффективным потреблением ресурсов.

Решение разработать собственное ПО для моделирования было принято ввиду ряда аспектов исследовательской работы, которые требуют создания и анализа большего объема данных. Для этого требуется автоматизировать существенную часть процесса исследования ввиду ограниченного времени и ресурсов. Помимо этого, трудоемкость на последующих этапах работы значительно снижается благодаря структурированности получаемой информации.
Были введены следующие требования к работе имитационной модели:

В рамках цели данной работы поставлены следующие задачи:
\begin{enumerate}
	\item Используя наработки по ранее разработанной имитационной модели, оптимизировать ее для масштабного проведения численных экспериментов.
	\item Реализовать метод вычисления асимптотического распределения вероятностей, использую дискретное преобразование Фурье. 
	\item Реализовать алгоритм вычисления вариации для интервалов между моментами поступлениями заявок MMPP.
	\item Разработать и реализовать вспомогательные программные инструменты для проведения численных экспериментов.
	\item Провести апробацию машинного обучения для вычисления вариации длин интервалов между моментами поступления заявок из выходящего процесса рассматриваемой модели системы.
\end{enumerate}

Работа содержит 48 страниц, 17 рисунков, 2 таблицы, 60 источников.

В первом разделе описывается модель рассматриваемой системы массового обслуживания. Во втором разделе рассматривается разработка и применение имитационной модели. В третьем разделе описаны новые методы вычисления характеристик работы системы. В четвертом разделе рассматривается применение методов машинного обучения для анализа работы рассматриваемой модели системы.

 \clearpage
 