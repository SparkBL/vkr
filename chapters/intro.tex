\section*{\centering ВВЕДЕНИЕ}
\addcontentsline{toc}{section}{Введение}
Во многих жизненных ситуациях человек сталкивается с необходимостью получить доступ к различным распределенным ресурсам. Ресурсы могут быть разного вида и происхождения --- касса в магазине, запись к врачу и многое другое. Помимо повседневной деятельности, такая необходимость в распределенных ресурсах возникает и в других более специализированных сферах жизнедеятельности и науке --- протоколы множественного сетевого доступа, автоматизированное производство. Во всех данных ситуациях при доступе к ресурсу существует очередь, которая может быть упорядоченной, либо неупорядоченной.

Теория массового обслуживания занимается изучением и математическим моделированием различных способов организации доступа к распределенному ресурсу \cite{nazarov2010theory}. Своим возникновением теория массового обслуживания обязана А.К. Эрлангу, который получил ряд базовых формул для исследования это области, решая практические задачи оптимизации систем телефонной связи \cite{erlang1909theory}. Терминология, используемая в теории массового обслуживания также взяла свое начало в практических задачах Эрланга ---  заявка, прибор, требование, буфер и другие.

Классическими моделями массового обслуживания являются модели с очередью и модели с отказами. В моделях с очередью заявка (требование), которая обратилась к прибору (ресурсу) и застала его занятым, встает в очередь для ожидания обслуживания. В моделях с отказами заявка при невозможности получить обслуживание теряется. Существует достаточно много модификаций и комбинаций таких моделей. 
Во второй половине ХХ века происходит бурное развитие телекоммуникационных сетей, что приводит к необходимости моделировать и проектировать сети передачи данных различной структуры и назначения. Для этого стали применяться модели систем массового обслуживания с повторными вызовами. В системах такого типа заявка, которая не может получить доступ к ресурсу повторяет попытку захвата ресурса через случайную задержку, а не встает в очередь. В таком случае, в рамках теории массового обслуживания, говорят, что заявка находится на орбите. В англоязычной литературе такие модели называются retrial queue (RQ), большой обзор работ по исследованию которых приведен в монографии \cite{artalejo2010accessible}. Основные принципы функционирования моделей систем с повторными вызовами изложены в \cite{jesus2008retrial,falin1997retrial}. Такие модели массового обслуживания возникают при стохастическом моделировании многих протоколов связи и локальных сетей. В частности, данные модели использовались при проектировании алгоритмов доступа и устранения заторов транспортных и канальных уровней модели OSI \cite{bjornstad2006traffic,kritzinger1986performance,olypher2010computer}, так как позволяют моделировать поведение системы при различных дисциплинах обслуживания и параметрах.

Помимо повторных вызовов, существуют ситуации (например, сценарий call--центра), когда обслуживающие единицы имеют возможность делать исходящие запросы на обслуживание в тот период времени, когда они находятся в простое. Эта модель организации известна как парная коммутация. В \cite{falin1979model} получены интегральные формулы для частичных производящих функций и явные выражения для ожидаемого значения некоторых характеристик производительности системы с повторными вызовами и двухсторонней связью (парной коммутацией) в предположении, что длительности входящих и исходящих вызовов соответствуют одинаковым распределениям вероятностей времени обслуживания. Однако на практике это предположение носит ограничительный характер, поскольку разные типы клиентов обычно демонстрируют разное поведение и, следовательно, у них должны быть разные потребности в обслуживании. В \cite{artalejo2010mean} использовали метод анализа среднего значения для получения некоторых ожидаемых значений, связанных со временем ожидания в очереди на повторное обращение с двусторонней связью и различным распределением времени обслуживания входящих и исходящих вызовов.

Особенность RQ--систем с повторными вызовами заключается в том, что в них представлены разные типы заявок, что порождает множество новых дисциплин обслуживания, что, в свою очередь, является мощным инструментом при проектировании и оптимизации систем с множественным случайным доступом к ресурсу.

Также, в современных телекоммуникационных сетях возникают точечные процессы с изменяющейся скоростью поступления заявок. Для моделирования таких процессов в рамках теории массового обслуживания используется процесс Пуассона с марковской модуляцией (MMPP) \cite{baiocchi1993steady,lapatin2019asymptotic}, потому что он имеет механизм для учета временной неоднородности скорости поступления заявок, но даёт аналитически поддающиеся обработке результаты организации очередей \cite{meier1987fitting}. По этой причине MMPP широко используется в исследованиях сети Интернет, в частности, при помощи MMPP в  \cite{muscariello2004mmpp} была построена модель трафика, которая точно аппроксимирует LRD (Long Range Dependence) характеристики трассировок интернет-трафика. С использованием понятий сеансов и потоков, предлагаемая модель MMPP имитирует реальное иерархическое поведение процесса генерации пакетов пользователями сети Интернет. Она позволяет генерировать трафик с желаемыми характеристиками с возможностью устанавливать несколько входных параметров, которые имеют четкий физический смысл. Результаты доказывают, что поведение трафика в очередях, генерируемого моделью MMPP согласуется с моделью, созданной реальными следами пакетов, собранных на граничном маршрутизаторе при различных сценариях и нагрузке.

Несмотря на большое количество исследований в рассматриваемой области \cite{artalejo2010mean,nazarov2017asymptotic,phung2019retrial,kulkarni1983queueing,paul2018retrial}, наименее изученной составляющей RQ--систем является выходящий поток заявок \cite{daley1976queueing}, иначе, -- заявки, покидающие систему по завершении обслуживания. Сведения о выходящем потоке крайне важны, так как в целом характеризуют работу системы и являются показательными в задачах, связанных с оптимизацией обслуживания. Усложняют анализ выходящего потока наличие разных типов заявок --- объем обслуженных заявок одного типа напрямую влияет на другой. При различных параметрах обслуживания результат работы системы и степень зависимости выходящих потоков заявок могут существенно разниться. 

В данной работе будут рассматриваться несколько моделей RQ--систем: с суммарных выходящим потоком и двумерным, а так же модель системы, интенсивность поступления заявок в котором периодически меняется. Целью данной работы является исследование корреляции в двумерном выходящем потоке модели узла обработки запросов с повторными обращениями, вызываемыми заявками и разными моделями входящего потока обращений. В рамках данной цели поставлены следующие задачи:
\begin{enumerate}
	\item Построить математические модели функционирования узла обработки запросов с повторными обращениями, вызываемыми заявками и различными входящими потоками обращений.
	\item Для предложенных моделей с помощью метода асимптотического анализа получить аналитические формулы аппроксимации двумерной характеристической функции числа обслуженных заявок входящего потока и обслуженных вызываемых требований. 
	\item На основании полученных аппроксимаций характеристической функции реализовать алгоритм вычисления распределения вероятностей, коэффициента корреляции и других числовых характеристик двумерного случайного процесса числа обслуженных заявок входящего потока и обслуженных вызываемых требований.
	\item Разработать имитационную модель предложенных математических моделей узлов обработки запросов.
	\item Реализовать предложенную имитационную модель математических моделей узлов обработки запросов.
	\item C помощью имитационной модели сделать оценку области применимости асимптотических результатов.
	\item Провести численный анализ коэффициента корреляции компонент двумерного выходящего потока предложенных математических моделей узла обработки запросов.
\end{enumerate}


 \clearpage