\section* {Введение}
\addcontentsline{toc}{section}{Введение}
Множество ситуаций обслуживания с использованием очередей имеют особенность, заключающуюся в том, что клиенты, обнаруживающие, что зона обслуживания занята по прибытии, присоединяются к группе неудовлетворенных клиентов, которые повторяют свой запрос через некоторое случайное время. В таком случае, в рамках теории массового обслуживания, говорят, что заявка находится на орбите.


Такие модели массового обслуживания возникают при стохастическом моделировании многих протоколов связи (стандарт IEEE 802.11), локальных сетей и повседневных жизненных ситуаций. Самый простой и очевидный пример - это человек, который звонит по телефону. Если линия занята, значит, он не может стоять в очереди, но через некоторое время снова испытывает удачу \cite{erlang1909theory}.  


Основные принципы систем массового обслуживания с повторными вызовами изложены в \cite{jesus2008retrial,falin1997retrial}, а так же в библиографической информации \cite{artalejo2010accessible}.

Помимо повторных вызовов, существуют ситуации (например, сценарий call-центра), когда обслуживающие единицы имеют возможность делать исходящие запросы на обслуживание в тот период времени, когда они находятся в простое. Эта модель организации известна как парная коммутация. В \cite{falin1979model} получены интегральные формулы для частичных производящих функций и явные выражения для ожидаемого значения некоторых характеристик производительности системы с повторными вызовами и двухсторонней связью в предположении, что длительности входящих и исходящих вызовов соответствуют одинаковым распределение времени обслуживания. Однако на практике это предположение носит ограничительный характер, поскольку разные типы клиентов обычно демонстрируют разное поведение и, следовательно, у них должны быть разные потребности в обслуживании.В \cite{artalejo2010mean} использовали метод анализа среднего значения для получения некоторых ожидаемых значений, связанных со временем ожидания в очереди на повторное обращение с двусторонней связью и различным распределением времени обслуживания входящих и исходящих вызовов. (абзац полностью слизан со статьи Phung Duc Two way communcication)







СТАТЬЯ\\
В этой работе мы рассматриваем двумерный выходящий поток системы массового обслуживания \cite{kendall1953stochastic, lapatin2019asymptotic} с повторными вызовами \cite {jesus2008retrial} и вызываемыми заявками \cite {kulkarni1983queueing}. Такую систему можно интерпретировать как узел обработки запросов с множественным произвольным доступом, который в свободное от обработки запросов время может запрашивать самодиагностику или любую другую процедуру, которая будет продолжаться в течение произвольного времени. Также рассматриваемая система может быть применена для моделирования узлов обработки с разными типами заявок. Заявки одного типа не теряются и будут обслуживаться в любом случае, а приложения другого типа будут обслуживаться только со свободным обслуживающей единицей.


В данной работе рассматривается влияние параметров системы на значения асимптотического коэффициента корреляции компонентов двумерного процесса вывода различных типов приложений. Для исследования системы используется метод асимптотического анализа для нахождения вида предельного двумерного распределения количества обслуженных заявок входящего потока и количества обслуженных вызванных заявок за некоторое время $t$, при условии, что на орбите \cite{nazarov2017asymptotic} наблюдается большая задержка заявок.


 \clearpage