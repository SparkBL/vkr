\section{Уравнения Колмогорова}
Итак, мы имеем три характеристики, определяющие результат функционирования системы за некоторое время $\textit{i(t)}$: состояние прибора – $\textit{k(t)}$, количество заявок на орбите – $\textit{i(t)}$, количество обслуженных заявок входящего потока – $m_{1}(t)$, количество обслуженных вызванных заявок – $m_{2}(t)$,  что можно представить в виде четырех-мерного Марковского процесса
\begin{equation*}
	\{k(t),i(t),m_{1}(t),m_{2}(t)\}
\end{equation*}
Заметим, что именно такая комбинация характеристик будет являться Марковским процессом, так как даёт достаточно информации о том, какое состояние система примет в следующий момент времени. Для этого необходимо знать, в каком состоянии прибор был в предшествующий момент времени, и какое количество заявок находилось в источнике повторных вызовов.
Следующее состояние, которое прибор может принять, зависит от состояния, в котором он находился прежде, то есть, каждое из трех состояний $\textit{k(t)}$ принимается прибором с некоторыми вероятностями. Введем их в рассмотрение
\begin{equation*}
	\begin{split}
	P\{k(t)=0,i(t)=i,m_{1}(t)=m_{1},m_{2}(t)=m_{2}\} &=P_{0}(i,m_{1},m_{2},t)\\
	P\{k(t)=1,i(t)=i,m_{1}(t)=m_{1},m_{2}(t)=m_{2}\} &=P_{1}(i,m_{1},m_{2},t)\\
	P\{k(t)=2,i(t)=i,m_{1}(t)=m_{1},m_{2}(t)=m_{2}\} &=P_{2}(i,m_{1},m_{2},t)
	\end{split}
	\end{equation*}
Запишем получившуюся систему уравнений
\begin{equation} \label{kolmogorov_equations}
\begin{split}
\frac{{\partial P_{0}(i,m_{1},m_{2},t)}}{{\partial t}} &= -(\lambda + i\sigma + \alpha)P_{0}(i,m_{1},m_{2},t) + P_{1}(i,m_{1}-1,m_{2},t)\mu_{1} +\\  &+ P_{2}(i,m_{1},m_{2}-1,t)\mu_{2} ,
\\
\frac{{\partial P_{1}(i,m_{1},m_{2},t)}}{{\partial t}} &= -(\lambda + \mu_{1})P_{1}(i,m_{1},m_{2},t) + (i+1)\sigma P_{0}(i+1,m_{1},m_{2},t) +\\ &+ \lambda  P_{0}(i,m_{1},m_{2},t),
\\
\frac{{\partial P_{2}(i,m_{1},m_{2},t)}}{{\partial t}} &= -(\lambda + \mu_{2})P_{2}(i,m_{1},m_{2},t) + \lambda P_{2}(i-1,m_{1},m_{2},t)  +\\ &+ \alpha  P_{0}(i,m_{1},m_{2},t).
\end{split}
\end{equation}	
Полученная система уравнений – система дифференциальных уравнений Колмогорова, где в левой части каждого уравнения находится производная вероятности состояния рассматриваемого процесса, а в правой – сумма произведений вероятностей состояний, из которых прибор может принять это состояние, на интенсивности соответствующих потоков заявок. Решением данной системы будут являться вероятности всех состояний прибора в виде функций времени. Таким образом, задача сводится к решению данной системы дифференциальных уравнений.
Решить данную систему аналитически не получится, так как это система бесконечного числа дифференциальных конечно-разностных уравнений с переменными коэффициентами. 
Для того, чтобы перейти к конечному числу уравнений, введем частные характеристические функции, обозначив $j=\sqrt{-1}$,
\begin{equation*}
	H_{k}(u,u_{1},u_{2},t) = \sum_{i=0}^{\infty}
	\sum_{m_{1}=0}^{\infty}
	\sum_{m_{2}=0}^{\infty}  
	e^{jui}e^{ju_{1}m_{1}}e^{ju_{2}m_{2}} P_{k}(i,m_{1},m_{2},t).
\end{equation*}
Тогда перепишем систему (\ref{kolmogorov_equations}) в виде
\begin{equation} \label{characteristic_equations}
	\begin{split}
		\frac{{\partial H_{0}(u,u_{1},u_{2},t)}}{{\partial t}} &= -(\lambda + \alpha)H_{0}(u,u_{1},u_{2},t) + j\sigma
		\frac{{\partial H_{0}(u,u_{1},u_{2},t)}}{{\partial u}} +\\  &+ \mu_{1} e^{ju_{1}}H_{1}(u,u_{1},u_{2},t) + \mu_{2}e^{ju_{2}}H_{2}(u,u_{1},u_{2},t) ,
		\\
		\frac{{\partial H_{1}(u,u_{1},u_{2},t)}}{{\partial t}} &= -(\lambda + \mu_{1})H_{1}(u,u_{1},u_{2},t) - j\sigma e^{-ju}
		\frac{{\partial H_{0}(u,u_{1},u_{2},t)}}{{\partial u}} +\\  &+ \lambda H_{0}(u,u_{1},u_{2},t) + \lambda e^{ju}H_{1}(u,u_{1},u_{2},t) ,
		\\
		\frac{{\partial H_{2}(u,u_{1},u_{2},t)}}{{\partial t}} &= -(\lambda + \mu_{2})H_{2}(u,u_{1},u_{2},t)  + \lambda e^{ju}H_{2}(u,u_{1},u_{2},t) +\\  &+ \alpha H_{0}(u,u_{1},u_{2},t).
	\end{split}
\end{equation}  
Таким образом, мы получили ровно три дифференциальных уравнения в частных производных с переменными коэффициентами.
\clearpage