\subsection{RQ-система с входящим MMPP-потоком и двумерным выходящим}
В случае RQ-системы с MMPP-потоком, к описанной ранее общей модели добавляется так же процесс $n_(t)$ - состояние входящего потока в момент времени $t$. Тогда, результат функционирования системы будет описываться пяти-мерным Марковским процессом
\begin{equation*}
	\{k(t),n(t),i(t),m_{1}(t),m_{2}(t)\}
\end{equation*}

\subsubsection{Уравнения Колмогорова}

Исходя из этого описанного Марковского процесса, вероятности того, что прибор будет находится в одном из трех состояний $k$, на орбите будет $i$ заявок, $m_{1}$ заявок входящего потока и $m_{2}$ вызванных заявок будет обслужено, а управляющая MMPP-потоком цепь Маркова будет находится в состоянии $n$, будут иметь вид
\begin{equation*}
	\begin{split}
		P\{k(t)=0,n(t)=n,i(t)=i,m_{1}(t)=m_{1},m_{2}(t)=m_{2}\} &=P_{0}(n,i,m_{1},m_{2},t)\\
		P\{k(t)=1,n(t)=n,i(t)=i,m_{1}(t)=m_{1},m_{2}(t)=m_{2}\} &=P_{1}(n,i,m_{1},m_{2},t)\\
		P\{k(t)=2,n(t)=n,i(t)=i,m_{1}(t)=m_{1},m_{2}(t)=m_{2}\} &=P_{2}(n,i,m_{1},m_{2},t)
	\end{split}
\end{equation*} 

На основе данных вероятностей запишем систему уравнений Колмогорова с учетом текущего состояния входящего потока
\begin{equation} \label{kolmogorov_equations_twodim_map}
	\begin{split}
		\frac{{\partial P_{0}(n,i,m_{1},m_{2},t)}}{{\partial t}} &= -(\lambda_{n} + i\sigma + \alpha)P_{0}(n,i,m_{1},m_{2},t) + P_{1}(n,i,m_{1}-1,m_{2},t)\mu_{1} +\\  &+ P_{2}(n,i,m_{1},m_{2}-1,t)\mu_{2} + \sum_{v=1}^{N} P_{0}(v,i,m_{1},m_{2},t)q_{vn},
		\\
		\frac{{\partial P_{1}(n,i,m_{1},m_{2},t)}}{{\partial t}} &= -(\lambda_{n} + \mu_{1})P_{1}(n,i,m_{1},m_{2},t) + (i+1)\sigma P_{0}(n,i+1,m_{1},m_{2},t) +\\ &+ \lambda_{n}  P_{0}(i,m_{1},m_{2},t) + \sum_{v=1}^{N}P_{1}(v,i,m_{1},m_{2},t)q_{vn},
		\\
		\frac{{\partial P_{2}(n,i,m_{1},m_{2},t)}}{{\partial t}} &= -(\lambda_{n} + \mu_{2})P_{2}(n,i,m_{1},m_{2},t) + \lambda_{n} P_{2}(n,i-1,m_{1},m_{2},t)  +\\ &+ \alpha  P_{0}(n,i,m_{1},m_{2},t) +  \sum_{v=1}^{N}P_{2}(v,i,m_{1},m_{2},t)q_{vn}.
	\end{split}
\end{equation}	


Аналогично процедуре решения вышеописанных систем, введем частные характеристические функции, обозначив $j=\sqrt{-1}$
\begin{equation*}
	H_{k}(n,u,u_{1},u_{2},t) = \sum_{i=0}^{\infty}
	\sum_{m_{1}=0}^{\infty}
	\sum_{m_{2}=0}^{\infty}  
	e^{jui}e^{ju_{1}m_{1}}e^{ju_{2}m_{2}} P_{k}(n,i,m_{1},m_{2},t).
\end{equation*}
Перепишем систему (\ref{kolmogorov_equations_twodim_map}) с учетом введенных частных характеристических функций
\begin{equation} \label{characteristic_equations_twodim_map}
	\begin{split}
		\frac{{\partial H_{0}(n,u,u_{1},u_{2},t)}}{{\partial t}} &= -(\lambda_{n} + \alpha)H_{0}(n,u,u_{1},u_{2},t) + j\sigma
		\frac{{\partial H_{0}(n,u,u_{1},u_{2},t)}}{{\partial u}} +\\  &+ \mu_{1} e^{ju_{1}}H_{1}(n,u,u_{1},u_{2},t) + \mu_{2}e^{ju_{2}}H_{2}(n,u,u_{1},u_{2},t) + \sum_{v=1}^{N}H_{0}(v,u,u_{1},u_{2},t)q_{vn} ,
		\\
		\frac{{\partial H_{1}(n,u,u_{1},u_{2},t)}}{{\partial t}} &= -(\lambda_{n} + \mu_{1})H_{1}(n,u,u_{1},u_{2},t) - j\sigma e^{-ju}
		\frac{{\partial H_{0}(n,u,u_{1},u_{2},t)}}{{\partial u}} +\\  &+ \lambda_{n} H_{0}(n,u,u_{1},u_{2},t) + \lambda_{n} e^{ju}H_{1}(n,u,u_{1},u_{2},t) + \sum_{v=1}^{N}H_{1}(v,u,u_{1},u_{2},t)q_{vn} ,
		\\
		\frac{{\partial H_{2}(n,u,u_{1},u_{2},t)}}{{\partial t}} &= -(\lambda_{n} + \mu_{2})H_{2}(n,u,u_{1},u_{2},t)  + \lambda_{n} e^{ju}H_{2}(n,u,u_{1},u_{2},t) +\\  &+ \alpha H_{0}(n,u,u_{1},u_{2},t) + \sum_{v=1}^{N}H_{2}(v,u,u_{1},u_{2},t)q_{vn}.
	\end{split}
\end{equation}  
Для дальнейшего анализа введем следующие обозначения
\begin{equation*}	
\boldsymbol{H}_{k}(u,u_{1},u_{2},t) = \{H_{k}(1,u,u_{1},u_{2},t),H_{k}(2,u,u_{1},u_{2},t),\dots,H_{k}(N,u,u_{1},u_{2},t)\},
\end{equation*}
а так же единичную матрицу $\boldsymbol{I}$ размера $N$.
Тогда система \ref{characteristic_equations_twodim_map} примет вид
\begin{equation} \label{characteristic_equations_twodim_map_matrix}
	\begin{split}
		\frac{{\partial \boldsymbol{H}_{0}(u,u_{1},u_{2},t)}}{{\partial t}} &= (\boldsymbol{Q}-\boldsymbol{\Lambda}-\alpha\boldsymbol{I})\boldsymbol{H}_{0}(u,u_{1},u_{2},t) + \mu_{1} e^{ju_{1}}\boldsymbol{H}_{1}(n,u,u_{1},u_{2},t)  + \\  &+ \mu_{2}e^{ju_{2}}\boldsymbol{H}_{2}(u,u_{1},u_{2},t) + j\sigma
		\frac{{\partial \boldsymbol{H}_{0}(u,u_{1},u_{2},t)}}{{\partial u}},
		\\
		\frac{{\partial \boldsymbol{H}_{1}(u,u_{1},u_{2},t)}}{{\partial t}} &= \boldsymbol{\Lambda} \boldsymbol{H}_{0}(u,u_{1},u_{2},t) +  (\boldsymbol{Q}+(e^{ju}-1)\boldsymbol{\Lambda} - \boldsymbol{I}\mu_{1})\boldsymbol{H}_{1}(u,u_{1},u_{2},t) -\\ &- j\sigma e^{-ju}
		\frac{{\partial \boldsymbol{H}_{0}(u,u_{1},u_{2},t)}}{{\partial u}},
		\\
		\frac{{\partial \boldsymbol{H}_{2}(u,u_{1},u_{2},t)}}{{\partial t}} &= \alpha \boldsymbol{H}_{0}(u,u_{1},u_{2},t) + (\boldsymbol{Q}+(e^{ju}-1)\boldsymbol{\Lambda} - \boldsymbol{I}\mu_{2})\boldsymbol{H}_{2}(u,u_{1},u_{2},t).
	\end{split}
\end{equation} 
\subsubsection{Метод асимптотического анализа}
Полученную систему дифференциальных уравнений (\ref{characteristic_equations_twodim_matrix}) будем решать методом асимптотического анализа в предельном условии большой задержки заявок на орбите ($\sigma \xrightarrow{} 0$).

Обозначим $\epsilon = \sigma,   u= \epsilon w,   \boldsymbol{F}_{k}(w,u_{1},u_{2},t,\epsilon) = \boldsymbol{F}_{k}(u,u_{1},u_{2},t)$, тогда система запишется в виде
\begin{equation} \label{asymptotic_equations_twodim_map}
	\begin{split}
		\frac{{\partial \boldsymbol{F}_{0}(w,u_{1},u_{2},t,\epsilon)}}{{\partial t}} &= (\boldsymbol{Q}-\boldsymbol{\Lambda}-\alpha\boldsymbol{I})\boldsymbol{F}_{0}(w,u_{1},u_{2},t,\epsilon) + \mu_{1} e^{ju_{1}}\boldsymbol{F}_{1}(w,u_{1},u_{2},t,\epsilon)  + \\  &+ \mu_{2}e^{ju_{2}}\boldsymbol{F}_{2}(w,u_{1},u_{2},t,\epsilon) + j
	\frac{{\partial \boldsymbol{F}_{0}(w,u_{1},u_{2},t,\epsilon)}}{{\partial w}},
	\\
	\frac{{\partial \boldsymbol{F}_{1}(w,u_{1},u_{2},t,\epsilon)}}{{\partial t}} &= \boldsymbol{\Lambda} \boldsymbol{F}_{0}(w,u_{1},u_{2},t,\epsilon) +  (\boldsymbol{Q}+(e^{j\epsilon w}-1)\boldsymbol{\Lambda} - \boldsymbol{I}\mu_{1})\boldsymbol{F}_{1}(w,u_{1},u_{2},t,\epsilon) -\\ &- j e^{-j\epsilon w}
	\frac{{\partial \boldsymbol{F}_{0}(w,u_{1},u_{2},t,\epsilon)}}{{\partial w}},
	\\
	\frac{{\partial \boldsymbol{F}_{2}(w,u_{1},u_{2},t,\epsilon)}}{{\partial t}} &= \alpha \boldsymbol{F}_{0}(w,u_{1},u_{2},t,\epsilon) + (\boldsymbol{Q}+(e^{j\epsilon w}-1)\boldsymbol{\Lambda} - \boldsymbol{I}\mu_{2})\boldsymbol{F}_{2}(w,u_{1},u_{2},t,\epsilon).
	\end{split}
\end{equation}  

Решение системы (\ref{asymptotic_equations_twodim_map}) будет сформулировано в теореме.
\begin{theorem}
	Асимптотические приближение двумерной характеристической функции числа обслуженных заявок входящего потока и числа обслуженных вызванных заявок за некоторое время $t$ имеет вид
	\begin{equation*} \label{theorem_twodim_map}
		\begin{split}
		  \lim_{\sigma \xrightarrow{} 0} M\{\exp(ju_{1}m_{1}(t))\exp(ju_{2}m_{2}(t))\} &= 
			 \lim_{\epsilon \xrightarrow{} 0} \{ \sum_{k=0}^{2}F_{k}(0,u_{1},u_{2},t,\epsilon) \}e = \boldsymbol{R} \cdot \exp\{G(u_{1},u_{2})t\}ee,
		\end{split}
	\end{equation*}
	где матрица $G(u_{1},u_{2})$ представима в виде
	\begin{equation*}
		\boldsymbol{G}(u_{1},u_{2})=\begin{bmatrix}
			\boldsymbol{Q}-\boldsymbol{\Lambda}-(\alpha + \kappa)\boldsymbol{I} & \mu_{1}e^{ju_{1}}\boldsymbol{I} &  \mu_{2}e^{ju_{2}}\boldsymbol{I}\\
			\boldsymbol{\Lambda}+\kappa\boldsymbol{I} & \boldsymbol{Q}-\mu_{1}\boldsymbol{I} & 0\\
			\alpha\boldsymbol{I} & 	0 &	\boldsymbol{Q}-\mu_{2}\boldsymbol{I}
		\end{bmatrix}^{T},
	\end{equation*}
	вектор-строка $\boldsymbol{R}=\{R_{0},R_{1},R_{2}\}$ - стационарное распределение вероятности случайного процесса $\{k(t),n(t)\}$, где, соответственно,  $\boldsymbol{R}_{k}$ имеет размерность $N$
	\begin{equation*}
		\boldsymbol{R}=\{\frac{\mu_{2}(\mu_{1} - \lambda)}{\mu_{1}(\mu_{2} - \alpha)},\frac{\lambda}{\mu_{1}},\frac{\alpha(\mu_{1} - \lambda)}{\mu_{1}(\mu_{2} + \alpha)}\},
	\end{equation*}
	$\kappa$ - нормированное среднее число заявок на орбите
	\begin{equation*}
		\kappa = \frac{\lambda(\lambda \mu_{2} + \alpha \mu_{1})}{\mu_{2}(\mu_{1} - \lambda)},
	\end{equation*}
	 а $e$ и $ee$ - единичные вектор столбцы размерности $N$ и $N \cdot K$ соответственно.
\end{theorem}
\begin{proof}
	Делая предельный переход $ \lim_{\epsilon \xrightarrow{} 0}\boldsymbol{F}_{k}(w,u_{1},u_{2},t,\epsilon) = \boldsymbol{F}_{k}(w,u_{1},u_{2},t)$  в полученной системе (\ref{asymptotic_equations_twodim_map}) , система уравнений будет записана в виде
	\begin{equation} \label{eps_limit_twodim_map}
		\begin{split}
			\frac{{\partial \boldsymbol{F}_{0}(w,u_{1},u_{2},t)}}{{\partial t}} &= (\boldsymbol{Q}-\boldsymbol{\Lambda}-\alpha\boldsymbol{I})\boldsymbol{F}_{0}(w,u_{1},u_{2},t) + \mu_{1} e^{ju_{1}}\boldsymbol{F}_{1}(w,u_{1},u_{2},t)  + \\  &+ \mu_{2}e^{ju_{2}}\boldsymbol{F}_{2}(w,u_{1},u_{2},t) + j
			\frac{{\partial \boldsymbol{F}_{0}(w,u_{1},u_{2},t)}}{{\partial w}},
			\\
			\frac{{\partial \boldsymbol{F}_{1}(w,u_{1},u_{2},t)}}{{\partial t}} &= \boldsymbol{\Lambda} \boldsymbol{F}_{0}(w,u_{1},u_{2},t) +  (\boldsymbol{Q} - \boldsymbol{I}\mu_{1})\boldsymbol{F}_{1}(w,u_{1},u_{2},t) -\\ &- j
			\frac{{\partial \boldsymbol{F}_{0}(w,u_{1},u_{2},t)}}{{\partial w}},
			\\
			\frac{{\partial \boldsymbol{F}_{2}(w,u_{1},u_{2},t)}}{{\partial t}} &= \alpha \boldsymbol{F}_{0}(w,u_{1},u_{2},t) + (\boldsymbol{Q} - \boldsymbol{I}\mu_{2})\boldsymbol{F}_{2}(w,u_{1},u_{2},t).
		\end{split}
	\end{equation}   
	Решение системы (\ref{eps_limit_twodim_map}) будет получено в следующей форме
	\begin{equation} \label{solution_twodim_map}
		\boldsymbol{F}_{k}(w,u_{1},u_{2},t) = \Phi(w)\boldsymbol{F}_{k}(u_{1},u_{2},t).
	\end{equation}  
	$\Phi(w)$ - асимптотическое приближение характеристической функции числа заявок на орбите при условии большой задержки на орбите.
	
	Подставив (\ref{solution_twodim_map}) в систему (\ref{eps_limit_twodim_map}) и разделив обе части уравнений на $\Phi(w)$, получим
	\begin{equation} \label{preresult_twodim_map}
	\begin{split}
		\frac{{\partial \boldsymbol{F}_{0}(u_{1},u_{2},t)}}{{\partial t}} &= (\boldsymbol{Q}-\boldsymbol{\Lambda}-\alpha\boldsymbol{I})\boldsymbol{F}_{0}(u_{1},u_{2},t) + \mu_{1} e^{ju_{1}}\boldsymbol{F}_{1}(u_{1},u_{2},t)  + \\  &+ \mu_{2}e^{ju_{2}}\boldsymbol{F}_{2}(u_{1},u_{2},t) + j\frac{\Phi'(w) }{\Phi(w)}
		 \boldsymbol{F}_{0}(u_{1},u_{2},t),
		\\
		\frac{{\partial \boldsymbol{F}_{1}(u_{1},u_{2},t)}}{{\partial t}} &= \boldsymbol{\Lambda} \boldsymbol{F}_{0}(u_{1},u_{2},t) +  (\boldsymbol{Q} - \boldsymbol{I}\mu_{1})\boldsymbol{F}_{1}(u_{1},u_{2},t) -\\ &- j\frac{\Phi'(w) }{\Phi(w)}
		 \boldsymbol{F}_{0}(u_{1},u_{2},t),
		\\
		\frac{{\partial \boldsymbol{F}_{2}(u_{1},u_{2},t)}}{{\partial t}} &= \alpha \boldsymbol{F}_{0}(u_{1},u_{2},t) + (\boldsymbol{Q} - \boldsymbol{I}\mu_{2})\boldsymbol{F}_{2}(u_{1},u_{2},t).
	\end{split}
	\end{equation}  
	Заметим, что $w$ содержится только в отношении $\frac{\Phi'(w) }{\Phi(w)}$, а остальные слагаемые и левые части уравнений не зависят от $w$. Это означает, что  $\Phi(w)$ имеет вид экспоненты. Учитывая, что  $\Phi(w)$ имеет смысл асимптотического приближения характеристической функции числа заявок на орбите, мы можем конкретизировать вид данной функции
	\begin{equation*}
		\frac{\Phi'(w) }{\Phi(w)} = \frac{e^{j\kappa w}j\kappa}{e^{j\kappa w}},
	\end{equation*} 
	где $\kappa$ - нормированное среднее число заявок на орбите, которое было получено в \cite{nazarov2017asymptotic} и имеет вид 
	\begin{equation*}
		\kappa = \frac{\lambda(\lambda \mu_{2} + \alpha \mu_{1})}{\mu_{2}(\mu_{1} - \lambda)}.
	\end{equation*}
	
	Исходя из этого, система (\ref{preresult_twodim_map}) примет следующий вид
	\begin{equation} \label{result_twodim_map}
		\begin{split}
			\frac{{\partial \boldsymbol{F}_{0}(u_{1},u_{2},t)}}{{\partial t}} &= (\boldsymbol{Q}-\boldsymbol{\Lambda}-(\alpha + \kappa)\boldsymbol{I})\boldsymbol{F}_{0}(u_{1},u_{2},t) + \mu_{1} e^{ju_{1}}\boldsymbol{F}_{1}(u_{1},u_{2},t)  + \\  &+ \mu_{2}e^{ju_{2}}\boldsymbol{F}_{2}(u_{1},u_{2},t),
			\\
			\frac{{\partial \boldsymbol{F}_{1}(u_{1},u_{2},t)}}{{\partial t}} &= (\boldsymbol{\Lambda} + \kappa\boldsymbol{I}) \boldsymbol{F}_{0}(u_{1},u_{2},t) +  (\boldsymbol{Q} - \boldsymbol{I}\mu_{1})\boldsymbol{F}_{1}(u_{1},u_{2},t) + \\&+ 0\boldsymbol{F}_{2}(u_{1},u_{2},t),
			\\
			\frac{{\partial \boldsymbol{F}_{2}(u_{1},u_{2},t)}}{{\partial t}} &= \alpha \boldsymbol{F}_{0}(u_{1},u_{2},t) + 0\boldsymbol{F}_{1}(u_{1},u_{2},t) +  (\boldsymbol{Q} - \boldsymbol{I}\mu_{2})\boldsymbol{F}_{2}(u_{1},u_{2},t).
		\end{split}
	\end{equation}  
	Введем следующие обозначения
	\begin{equation*}
		\boldsymbol{F}(u_{1},u_{2},t) = \{F_{0}(u_{1},u_{2},t),F_{1}(u_{1},u_{2},t),F_{2}(u_{1},u_{2},t)\},
	\end{equation*}  
	\begin{equation*}
		\boldsymbol{G}(u_{1},u_{2})=\begin{bmatrix}
			\boldsymbol{Q}-\boldsymbol{\Lambda}-(\alpha + \kappa)\boldsymbol{I} & \mu_{1}e^{ju_{1}}\boldsymbol{I} &  \mu_{2}e^{ju_{2}}\boldsymbol{I}\\
			\boldsymbol{\Lambda}+\kappa\boldsymbol{I} & \boldsymbol{Q}-\mu_{1}\boldsymbol{I} & 0\\
			\alpha\boldsymbol{I} & 	0 &	\boldsymbol{Q}-\mu_{2}\boldsymbol{I}
		\end{bmatrix}^{T},
	\end{equation*}
	$\boldsymbol{G}(u_{1},u_{2})$ - транспонированная матрица коэффициентов системы (\ref{result_twodim_map}).
	Тогда получим следующее матричное уравнение
	\begin{equation*}
		\frac{{\partial \boldsymbol{F}(u_{1},u_{2},t)}}{{\partial t}} =\boldsymbol{F}(u_{1},u_{2},t)\boldsymbol{G}(u_{1},u_{2}),
	\end{equation*}
	общее решение которого имеет вид
	\begin{equation} \label{diff_twodim_map}
		\boldsymbol{F}(u_{1},u_{2},t)=\boldsymbol{C}e^{\boldsymbol{G}(u_{1},u_{2})t}.
	\end{equation}
	Для того, чтобы получить единственное решение, которое соответствует поведению рассматриваемой системы, примем в рассмотрение начальное условие
	\begin{equation} \label{cauchi_condition_twodim_map}
		\boldsymbol{F}(u_{1},u_{2},0)=\boldsymbol{R},
	\end{equation}
	где вектор-строка $\boldsymbol{R}$ - стационарное распределение вероятности состояния прибора, то есть процесса $k(t)$, которое имеет форму \cite{nazarov2017asymptotic}
	\begin{equation*}
		\boldsymbol{R}=\{\frac{\mu_{2}(\mu_{1} - \lambda)}{\mu_{1}(\mu_{2} - \alpha)},\frac{\lambda}{\mu_{1}},\frac{\alpha(\mu_{1} - \lambda)}{\mu_{1}(\mu_{2} + \alpha)}\}.
	\end{equation*}
	Описав начальное условие, мы можем перейти к решению задачи Коши (\ref{diff_twodim}, \ref{cauchi_condition_twodim_map}).
	
	Поскольку нас интересует распределение вероятностей количества заявок в выходных процессах, необходимо найти маргинальное распределение. Для этого суммируем компоненты вектор-строки $\boldsymbol{F}(u_{1},u_{2},t)$ по $k$ и умножаем результат на единичный вектор-столбец $\boldsymbol{E}$. Получим
	\begin{equation}\label{approximation_twodim_map}
		\boldsymbol{F}(u_{1},u_{2},t)\boldsymbol{E}=\boldsymbol{R}e^{\boldsymbol{G}(u_{1},u_{2})t}\boldsymbol{E}.
	\end{equation}
	Эта формула позволяет найти асимптотическое приближение характеристической функции количества вызванных и входящих заявок, обслуженных системой к некоторому моменту времени $t$. Другими словами, формула (\ref{approximation_twodim_map}) является решением рассматриваемой системы. 
\end{proof}
\clearpage