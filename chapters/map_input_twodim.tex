\subsection{RQ--система с входящим MMPP и двумерным выходящим потоком} \label{section_map_twodim}
В случае RQ--системы с MMPP--потоком, к описанной ранее общей модели добавляется так же процесс $n_(t)$ --- состояние входящего потока в момент времени $t$. Тогда, результат функционирования системы будет описываться пяти--мерным Марковским процессом
\begin{equation*}
	\{k(t),n(t),i(t),m_{1}(t),m_{2}(t)\}
\end{equation*}

\subsubsection{Уравнения Колмогорова}

Исходя из описанного Марковского процесса, вероятности того, что прибор будет находится в одном из трех состояний $k$, на орбите будет $i$ заявок, $m_{1}$ заявок входящего потока и $m_{2}$ вызванных заявок будет обслужено, а управляющая MMPP--потоком цепь Маркова будет находится в состоянии $n$, будут иметь вид
\begin{equation*}
	\begin{split}
		P\{k(t)=0,n(t)=n,i(t)=i,m_{1}(t)=m_{1},m_{2}(t)=m_{2}\} &=P_{0}(n,i,m_{1},m_{2},t)\\
		P\{k(t)=1,n(t)=n,i(t)=i,m_{1}(t)=m_{1},m_{2}(t)=m_{2}\} &=P_{1}(n,i,m_{1},m_{2},t)\\
		P\{k(t)=2,n(t)=n,i(t)=i,m_{1}(t)=m_{1},m_{2}(t)=m_{2}\} &=P_{2}(n,i,m_{1},m_{2},t)
	\end{split}
\end{equation*} 

На основе данных вероятностей запишем систему уравнений Колмогорова с учетом текущего состояния входящего потока
\begin{equation} \label{kolmogorov_equations_twodim_map}
	\begin{split}
		\frac{{\partial P_{0}(n,i,m_{1},m_{2},t)}}{{\partial t}} &= -(\lambda_{n} + i\sigma + \alpha)P_{0}(n,i,m_{1},m_{2},t) + P_{1}(n,i,m_{1}-1,m_{2},t)\mu_{1} +\\  &+ P_{2}(n,i,m_{1},m_{2}-1,t)\mu_{2} + \sum_{v=1}^{N} P_{0}(v,i,m_{1},m_{2},t)q_{vn},
		\\
		\frac{{\partial P_{1}(n,i,m_{1},m_{2},t)}}{{\partial t}} &= -(\lambda_{n} + \mu_{1})P_{1}(n,i,m_{1},m_{2},t) + (i+1)\sigma P_{0}(n,i+1,m_{1},m_{2},t) +\\ &+ \lambda_{n}  P_{0}(i,m_{1},m_{2},t) + \sum_{v=1}^{N}P_{1}(v,i,m_{1},m_{2},t)q_{vn},
		\\
		\frac{{\partial P_{2}(n,i,m_{1},m_{2},t)}}{{\partial t}} &= -(\lambda_{n} + \mu_{2})P_{2}(n,i,m_{1},m_{2},t) + \lambda_{n} P_{2}(n,i-1,m_{1},m_{2},t)  +\\ &+ \alpha  P_{0}(n,i,m_{1},m_{2},t) +  \sum_{v=1}^{N}P_{2}(v,i,m_{1},m_{2},t)q_{vn}.
	\end{split}
\end{equation}	

Введем частные характеристические функции, обозначив $j=\sqrt{-1}$
\begin{equation*}
	H_{k}(n,u,u_{1},u_{2},t) = \sum_{i=0}^{\infty}
	\sum_{m_{1}=0}^{\infty}
	\sum_{m_{2}=0}^{\infty}  
	e^{jui}e^{ju_{1}m_{1}}e^{ju_{2}m_{2}} P_{k}(n,i,m_{1},m_{2},t).
\end{equation*}
Перепишем систему \eqref{kolmogorov_equations_twodim_map} с учетом введенных частных характеристических функций
\begin{equation} \label{characteristic_equations_twodim_map}
	\begin{split}
		\frac{{\partial H_{0}(n,u,u_{1},u_{2},t)}}{{\partial t}} &= -(\lambda_{n} + \alpha)H_{0}(n,u,u_{1},u_{2},t) + j\sigma
		\frac{{\partial H_{0}(n,u,u_{1},u_{2},t)}}{{\partial u}} +\\  &+ \mu_{1} e^{ju_{1}}H_{1}(n,u,u_{1},u_{2},t) + \mu_{2}e^{ju_{2}}H_{2}(n,u,u_{1},u_{2},t) +\\  &+ \sum_{v=1}^{N}H_{0}(v,u,u_{1},u_{2},t)q_{vn} ,
		\\
		\frac{{\partial H_{1}(n,u,u_{1},u_{2},t)}}{{\partial t}} &= -(\lambda_{n} + \mu_{1})H_{1}(n,u,u_{1},u_{2},t) - j\sigma e^{-ju}
		\frac{{\partial H_{0}(n,u,u_{1},u_{2},t)}}{{\partial u}} +\\  &+ \lambda_{n} H_{0}(n,u,u_{1},u_{2},t) + \lambda_{n} e^{ju}H_{1}(n,u,u_{1},u_{2},t) +\\  &+ \sum_{v=1}^{N}H_{1}(v,u,u_{1},u_{2},t)q_{vn} ,
		\\
		\frac{{\partial H_{2}(n,u,u_{1},u_{2},t)}}{{\partial t}} &= -(\lambda_{n} + \mu_{2})H_{2}(n,u,u_{1},u_{2},t)  + \lambda_{n} e^{ju}H_{2}(n,u,u_{1},u_{2},t) +\\  &+ \alpha H_{0}(n,u,u_{1},u_{2},t) + \sum_{v=1}^{N}H_{2}(v,u,u_{1},u_{2},t)q_{vn}.
	\end{split}
\end{equation}  
Для дальнейшего анализа введем следующие обозначения
\begin{equation*}	
\boldsymbol{H}_{k}(u,u_{1},u_{2},t) = \{H_{k}(1,u,u_{1},u_{2},t),H_{k}(2,u,u_{1},u_{2},t),\dots,H_{k}(N,u,u_{1},u_{2},t)\},
\end{equation*}
а так же диагональную единичную матрицу $\boldsymbol{I}$ размера $N$.
Тогда система \eqref{characteristic_equations_twodim_map} примет вид
\begin{equation} \label{characteristic_equations_twodim_map_matrix}
	\begin{split}
		\frac{{\partial \boldsymbol{H}_{0}(u,u_{1},u_{2},t)}}{{\partial t}} &= (\boldsymbol{Q}-\boldsymbol{\Lambda}-\alpha\boldsymbol{I})\boldsymbol{H}_{0}(u,u_{1},u_{2},t) + \mu_{1} e^{ju_{1}}\boldsymbol{H}_{1}(n,u,u_{1},u_{2},t)  + \\  &+ \mu_{2}e^{ju_{2}}\boldsymbol{H}_{2}(u,u_{1},u_{2},t) + j\sigma
		\frac{{\partial \boldsymbol{H}_{0}(u,u_{1},u_{2},t)}}{{\partial u}},
		\\
		\frac{{\partial \boldsymbol{H}_{1}(u,u_{1},u_{2},t)}}{{\partial t}} &= \boldsymbol{\Lambda} \boldsymbol{H}_{0}(u,u_{1},u_{2},t) +  (\boldsymbol{Q}+(e^{ju}-1)\boldsymbol{\Lambda} - \boldsymbol{I}\mu_{1})\boldsymbol{H}_{1}(u,u_{1},u_{2},t) -\\ &- j\sigma e^{-ju}
		\frac{{\partial \boldsymbol{H}_{0}(u,u_{1},u_{2},t)}}{{\partial u}},
		\\
		\frac{{\partial \boldsymbol{H}_{2}(u,u_{1},u_{2},t)}}{{\partial t}} &= \alpha \boldsymbol{H}_{0}(u,u_{1},u_{2},t) + (\boldsymbol{Q}+(e^{ju}-1)\boldsymbol{\Lambda} - \boldsymbol{I}\mu_{2})\boldsymbol{H}_{2}(u,u_{1},u_{2},t).
	\end{split}
\end{equation} 
\subsubsection{Метод асимптотического анализа}
Полученную систему дифференциальных уравнений \eqref{characteristic_equations_twodim_map_matrix} будем решать методом асимптотического анализа в предельном условии большой задержки заявок на орбите ($\sigma \xrightarrow{} 0$).

Обозначим $\epsilon = \sigma,   u= \epsilon w,   \boldsymbol{F}_{k}(w,u_{1},u_{2},t,\epsilon) = \boldsymbol{H}_{k}(u,u_{1},u_{2},t)$, тогда система запишется в виде
\begin{equation} \label{asymptotic_equations_twodim_map}
	\begin{split}
		\frac{{\partial \boldsymbol{F}_{0}(w,u_{1},u_{2},t,\epsilon)}}{{\partial t}} &= (\boldsymbol{Q}-\boldsymbol{\Lambda}-\alpha\boldsymbol{I})\boldsymbol{F}_{0}(w,u_{1},u_{2},t,\epsilon) + \mu_{1} e^{ju_{1}}\boldsymbol{F}_{1}(w,u_{1},u_{2},t,\epsilon)  + \\  &+ \mu_{2}e^{ju_{2}}\boldsymbol{F}_{2}(w,u_{1},u_{2},t,\epsilon) + j
	\frac{{\partial \boldsymbol{F}_{0}(w,u_{1},u_{2},t,\epsilon)}}{{\partial w}},
	\\
	\frac{{\partial \boldsymbol{F}_{1}(w,u_{1},u_{2},t,\epsilon)}}{{\partial t}} &= \boldsymbol{\Lambda} \boldsymbol{F}_{0}(w,u_{1},u_{2},t,\epsilon) +  (\boldsymbol{Q}+(e^{j\epsilon w}-1)\boldsymbol{\Lambda} - \boldsymbol{I}\mu_{1})\cdot \\ &\cdot \boldsymbol{F}_{1}(w,u_{1},u_{2},t,\epsilon) - j e^{-j\epsilon w}
	\frac{{\partial \boldsymbol{F}_{0}(w,u_{1},u_{2},t,\epsilon)}}{{\partial w}},
	\\
	\frac{{\partial \boldsymbol{F}_{2}(w,u_{1},u_{2},t,\epsilon)}}{{\partial t}} &= \alpha \boldsymbol{F}_{0}(w,u_{1},u_{2},t,\epsilon) + (\boldsymbol{Q}+(e^{j\epsilon w}-1)\boldsymbol{\Lambda} - \boldsymbol{I}\mu_{2})\boldsymbol{F}_{2}(w,u_{1},u_{2},t,\epsilon).
	\end{split}
\end{equation}  

Решение системы \eqref{asymptotic_equations_twodim_map} будет сформулировано в последующих теоремах.

\begin{theorem} \label{R_theorem}
	Пусть $i(t)$ --- количество заявок на орбите в системе с входящим MMPP--потоком и вызываемыми заявками, тогда в стационарном режиме мы получим
	\begin{equation*}
		\lim_{\epsilon \xrightarrow{} 0}\{\sum_{k=0}^{2}F_{k}(w,0,0,t,\epsilon)\} = \lim_{\sigma \xrightarrow{} 0} M e^{jw\sigma i(t)} = e^{jw\kappa},
	\end{equation*}
где $\kappa$ --- положительный корень уравнения
\begin{equation*}
	\kappa \boldsymbol{R}_{0}(\kappa)\boldsymbol{e} = [\boldsymbol{R}_{1}(\kappa) + \boldsymbol{R}_{2}(\kappa)]\boldsymbol{\Lambda}\boldsymbol{e}.
\end{equation*}
Более того, векторы $\boldsymbol{R}_{k}$ определяются следующим образом
\begin{equation*}
	\left\{
	\begin{aligned}
		\boldsymbol{R}_{0}(\kappa) & = \boldsymbol{r}\{\boldsymbol{I} + [\boldsymbol{\Lambda} + \kappa\boldsymbol{I}](\mu_{1}\boldsymbol{I}-\boldsymbol{Q})^{-1}+\alpha(\mu_{2}\boldsymbol{I}-\boldsymbol{Q})^{-1}\}^{-1},\\
		\boldsymbol{R}_{1}(\kappa) & = \boldsymbol{R}_{0}(\kappa)[\boldsymbol{\Lambda} + \kappa\boldsymbol{I}](\mu_{1}\boldsymbol{I} - \boldsymbol{Q})^{-1},\\
		\boldsymbol{R}_{2}(\kappa) & = \alpha\boldsymbol{R}_{0}(\kappa)(\mu_{2}\boldsymbol{I} - \boldsymbol{Q})^{-1}
	\end{aligned}
\right.
\end{equation*}
Вектор--строка $\boldsymbol{r}$ --- стационарное распределение вероятности процесса управляющей цепи MMPP--потока $n(t)$, которое задается как единственное решение системы $\boldsymbol{r}\boldsymbol{Q} =0, \boldsymbol{r}\boldsymbol{e} = 1$.
\end{theorem}
\begin{proof}
	В системе уравнений \eqref{asymptotic_equations_twodim_map} мы обозначили $u_{1} = u_{2} = 0$, тем самым убирая процессы $m_{1}(t)$ и $m_{2}(t)$ из рассмотрения. Тогда, мы можем получить систему уравнений уже для трехмерного процесса $\{n(t),k(t),i(t)\}$, рассматривая его в стационарном режиме, что позволяет нам избавиться от производных по времени $t$.
	Обозначим 
	\begin{equation*}
		\boldsymbol{F}_{k}(w,\epsilon) = \lim_{t \xrightarrow{} \infty} \boldsymbol{F}_{k}(w,0,0,t,\epsilon),
	\end{equation*}
для получения следующей системы
\begin{equation}
		 \label{asymptotic_equations_twodim_map_no_time}
		 \begin{split}
		 &(\boldsymbol{Q}-\boldsymbol{\Lambda}-\alpha\boldsymbol{I})\boldsymbol{F}_{0}(w,\epsilon) + \mu_{1} \boldsymbol{F}_{1}(w,\epsilon)  +  \mu_{2}\boldsymbol{F}_{2}(w,\epsilon) + j
		 \boldsymbol{F}_{0}'(w,\epsilon)  = 0,
		\\
		 &\boldsymbol{\Lambda} \boldsymbol{F}_{0}(w,\epsilon) +  (\boldsymbol{Q}+(e^{j\epsilon w}-1)\boldsymbol{\Lambda} - \boldsymbol{I}\mu_{1})\boldsymbol{F}_{1}(w,\epsilon) - j e^{-j\epsilon w}
		 \boldsymbol{F}_{0}'(w,\epsilon)  = 0,
		\\
		&\alpha \boldsymbol{F}_{0}(w,\epsilon) + (\boldsymbol{Q}+(e^{j\epsilon w}-1)\boldsymbol{\Lambda} - \boldsymbol{I}\mu_{2})\boldsymbol{F}_{2}(w,\epsilon)  = 0.
	\end{split}
\end{equation}  
Делая в системе \eqref{asymptotic_equations_twodim_map_no_time} предельный переход $\epsilon \xrightarrow{} 0$, получим
\begin{equation}
	\label{asymptotic_equations_twodim_map_no_limit}
	\begin{split}
		&(\boldsymbol{Q}-\boldsymbol{\Lambda}-\alpha\boldsymbol{I})\boldsymbol{F}_{0}(w) + \mu_{1} \boldsymbol{F}_{1}(w)  +  \mu_{2}\boldsymbol{F}_{2}(w) + j
		\boldsymbol{F}_{0}'(w)  = 0,
		\\
		&\boldsymbol{\Lambda} \boldsymbol{F}_{0}(w) +  (\boldsymbol{Q} - \boldsymbol{I}\mu_{1})\boldsymbol{F}_{1}(w) - j
		\boldsymbol{F}_{0}'(w)  = 0,
		\\
		&\alpha \boldsymbol{F}_{0}(w) + (\boldsymbol{Q} - \boldsymbol{I}\mu_{2})\boldsymbol{F}_{2}(w)  = 0.
	\end{split}
\end{equation}
Решение данной системы будет найдено в следующей форме
\begin{equation} \label{R_solution}
	\boldsymbol{F}_{k} = \Phi(w)\boldsymbol{R}_{k},
\end{equation}
где $\boldsymbol{R}_{n}$ --- стационарное распределение вероятности прибора, а $\Phi(w)$ --- асимптотическое приближение характеристической функции числа заявок на орбите при условии большой задержки на орбите. Подставляя \eqref{R_solution} в \eqref{asymptotic_equations_twodim_map_no_limit} и деля на $\Phi(w)$, получим
\begin{equation}
	\label{asymptotic_equations_twodim_map_R}
	\begin{split}
		&(\boldsymbol{Q}-\boldsymbol{\Lambda}-\alpha\boldsymbol{I})\boldsymbol{R}_{0} + \mu_{1} \boldsymbol{R}_{1}  +  \mu_{2}\boldsymbol{R}_{2} + j
		\frac{\Phi'(w) }{\Phi(w)}\boldsymbol{R}_{0}  = 0,
		\\
		&\boldsymbol{\Lambda} \boldsymbol{R}_{0} +  (\boldsymbol{Q} - \boldsymbol{I}\mu_{1})\boldsymbol{R}_{1} - j\frac{\Phi'(w) }{\Phi(w)}
		\boldsymbol{R}_{0}  = 0,
		\\
		&\alpha \boldsymbol{R}_{0} + (\boldsymbol{Q} - \boldsymbol{I}\mu_{2})\boldsymbol{R}_{2}  = 0.
	\end{split}
\end{equation}
Вид функции $\Phi(w)$ уже ранее был нами конкретизирован в \eqref{Phi_concrete} так, что $j\frac{\Phi(w)'}{\Phi(w)} = -\kappa$. Подставим данное выражение в систему \eqref{asymptotic_equations_twodim_map_R}, получим
\begin{equation}
	\label{asymptotic_equations_twodim_map_R_final}
	\begin{split}
		&(\boldsymbol{Q}-\boldsymbol{\Lambda}-\alpha\boldsymbol{I})\boldsymbol{R}_{0} + \mu_{1} \boldsymbol{R}_{1}  +  \mu_{2}\boldsymbol{R}_{2} -\kappa{R}_{0}  = 0,
		\\
		&\boldsymbol{\Lambda} \boldsymbol{R}_{0} +  (\boldsymbol{Q} - \boldsymbol{I}\mu_{1})\boldsymbol{R}_{1} + \kappa
		\boldsymbol{R}_{0}  = 0,
		\\
		&\alpha \boldsymbol{R}_{0} + (\boldsymbol{Q} - \boldsymbol{I}\mu_{2})\boldsymbol{R}_{2}  = 0.
	\end{split}
\end{equation}
Запишем условия нормировки для стационарного распределения вероятностей числа обслуженных заявок прибором
\begin{equation*}
	\boldsymbol{R}_{0} + \boldsymbol{R}_{1} + \boldsymbol{R}_{2} = \boldsymbol{r}.
\end{equation*}
Основываясь на этом уравнении, а также на двух последних уравнения системы \eqref{asymptotic_equations_twodim_map_R_final} запишем систему
\begin{equation} \label{R_system}
	\left\{
	\begin{aligned}
		\boldsymbol{R}_{1}& = \boldsymbol{R}_{0}[\boldsymbol{\Lambda} + \kappa\boldsymbol{I}](\mu_{1}\boldsymbol{I} - \boldsymbol{Q})^{-1},\\
		\boldsymbol{R}_{2}& = \alpha\boldsymbol{R}_{0}(\mu_2\boldsymbol{I} - \boldsymbol{Q})^{-1},\\
		\boldsymbol{R}_{0}& + \boldsymbol{R}_{1} + \boldsymbol{R}_{2} = \boldsymbol{r}.
	\end{aligned}
	\right.
\end{equation}
Просуммируем уравнения системы \eqref{asymptotic_equations_twodim_map_no_time}
\begin{equation*}
	\begin{split}
		[\boldsymbol{F}_{0}(w,\epsilon) + \boldsymbol{F}_{1}(w,\epsilon) +  \boldsymbol{F}_{2}(w,\epsilon)]\boldsymbol{Q} &+\\  + 
		\boldsymbol{F}_{1}(w,\epsilon)(e^{jw\epsilon} - 1)\boldsymbol{\Lambda} + & \boldsymbol{F}_{2}(w,\epsilon)(e^{jw\epsilon} - 1)\boldsymbol{\Lambda} + je^{-jw\epsilon}(e^{jw\epsilon} - 1)\boldsymbol{F}_{0}'(w,\epsilon) = 0.
	\end{split}
\end{equation*}
Умножая получившееся уравнения на единичный вектор столбец $\boldsymbol{e}$, получим
\begin{equation*}
	\{\boldsymbol{F}_{1}(w,\epsilon) + \boldsymbol{F}_{2}(w,\epsilon)\}\boldsymbol{\Lambda}\boldsymbol{e} + je^{-jw\epsilon}\boldsymbol{F}_{0}'(w,\epsilon)\boldsymbol{e} = 0
\end{equation*}
Затем, подставим произведение \eqref{R_solution} в получившееся уравнение
\begin{equation*}
	[\boldsymbol{R}_{1} + \boldsymbol{R}_{2}]\boldsymbol{\Lambda}\boldsymbol{e} + j\frac{\Phi'(w)}{\Phi(w)}\boldsymbol{R}_{0}\boldsymbol{e} = 0
\end{equation*}
 и делаем замену
 \begin{equation} \label{R_kappa_exression}
 	[\boldsymbol{R}_{1} + \boldsymbol{R}_{2}]\boldsymbol{\Lambda}\boldsymbol{e} -\kappa\boldsymbol{R}_{0}\boldsymbol{e} = 0
\end{equation}
Из \eqref{R_kappa_exression} мы можем выразить $\kappa$ с помощью $\boldsymbol{R}_{0}$,$\boldsymbol{R}_{1}$ и $\boldsymbol{R}_{2}$. Помимо этого, мы можем переписать систему \eqref{R_system} в следующим виде
\begin{equation*}
	\left\{
	\begin{aligned}
		\boldsymbol{R}_{0}(\kappa) & = \boldsymbol{r}\{\boldsymbol{I} + [\boldsymbol{\Lambda} + \kappa\boldsymbol{I}](\mu_{1}\boldsymbol{I}-\boldsymbol{Q})^{-1}+\alpha(\mu_{2}\boldsymbol{I}-\boldsymbol{Q})^{-1}\}^{-1},\\
		\boldsymbol{R}_{1}(\kappa) & = \boldsymbol{R}_{0}(\kappa)[\boldsymbol{\Lambda} + \kappa\boldsymbol{I}](\mu_{1}\boldsymbol{I} - \boldsymbol{Q})^{-1},\\
		\boldsymbol{R}_{2}(\kappa) & = \alpha\boldsymbol{R}_{0}(\kappa)(\mu_{2}\boldsymbol{I} - \boldsymbol{Q})^{-1}
	\end{aligned}
	\right.
\end{equation*}
\end{proof}
Теорема \ref{R_theorem} является вспомогательной, так как основное решение рассматриваемой системы изложено в теореме \ref{mmpp_theorem} и нуждается в полученных на данном этапе результатах, а именно --- среднее число заявок на орбите при условии их большой задержки $\kappa$ и стационарное распределение вероятностей состояния прибора $\boldsymbol{R}_{k}$.

\begin{theorem} \label{mmpp_theorem}
	Асимптотические приближение двумерной характеристической функции числа обслуженных заявок входящего MMPP-потока и числа обслуженных вызванных заявок за некоторое время $t$ имеет вид
	\begin{equation*} \label{theorem_twodim_map}
		\begin{split}
		  \lim_{\sigma \xrightarrow{} 0} M\{\exp(ju_{1}m_{1}(t))\exp(ju_{2}m_{2}(t))\} &= 
			 \lim_{\epsilon \xrightarrow{} 0} \{ \sum_{k=0}^{2}F_{k}(0,u_{1},u_{2},t,\epsilon) \}\boldsymbol{e} =\\  &= \boldsymbol{R} \cdot \exp\{G(u_{1},u_{2})t\}\boldsymbol{ee},
		\end{split}
	\end{equation*}
	где матрица $\boldsymbol{G}(u_{1},u_{2})$ представима в виде
	\begin{equation*}
		\boldsymbol{G}(u_{1},u_{2})=\begin{bmatrix}
			\boldsymbol{Q}-\boldsymbol{\Lambda}-(\alpha + \kappa)\boldsymbol{I} & \mu_{1}e^{ju_{1}}\boldsymbol{I} &  \mu_{2}e^{ju_{2}}\boldsymbol{I}\\
			\boldsymbol{\Lambda}+\kappa\boldsymbol{I} & \boldsymbol{Q}-\mu_{1}\boldsymbol{I} & 0\\
			\alpha\boldsymbol{I} & 	0 &	\boldsymbol{Q}-\mu_{2}\boldsymbol{I}
		\end{bmatrix}^{T},
	\end{equation*}
	вектор--строка $\boldsymbol{R}=\{\boldsymbol{R}_{0},\boldsymbol{R}_{1},\boldsymbol{R}_{2}\}$ -- двумерное стационарное распределение вероятности случайного процесса $\{k(t),n(t)\}$, где, соответственно,  $\boldsymbol{R}_{k}$ имеет размерность $N$, $\kappa$ --- нормированное среднее число заявок на орбите, а $e$ и $ee$ --- единичные вектор--столбцы размерности $N$ и $N \cdot K$ соответственно.
\end{theorem}
\begin{proof}
	Делая предельный переход $ \lim_{\epsilon \xrightarrow{} 0}\boldsymbol{F}_{k}(w,u_{1},u_{2},t,\epsilon) = \boldsymbol{F}_{k}(w,u_{1},u_{2},t)$  в полученной системе \eqref{asymptotic_equations_twodim_map}, система уравнений будет записана в виде
	\begin{equation} \label{eps_limit_twodim_map}
		\begin{split}
			\frac{{\partial \boldsymbol{F}_{0}(w,u_{1},u_{2},t)}}{{\partial t}} &= (\boldsymbol{Q}-\boldsymbol{\Lambda}-\alpha\boldsymbol{I})\boldsymbol{F}_{0}(w,u_{1},u_{2},t) + \mu_{1} e^{ju_{1}}\boldsymbol{F}_{1}(w,u_{1},u_{2},t)  + \\  &+ \mu_{2}e^{ju_{2}}\boldsymbol{F}_{2}(w,u_{1},u_{2},t) + j
			\frac{{\partial \boldsymbol{F}_{0}(w,u_{1},u_{2},t)}}{{\partial w}},
			\\
			\frac{{\partial \boldsymbol{F}_{1}(w,u_{1},u_{2},t)}}{{\partial t}} &= \boldsymbol{\Lambda} \boldsymbol{F}_{0}(w,u_{1},u_{2},t) +  (\boldsymbol{Q} - \boldsymbol{I}\mu_{1})\boldsymbol{F}_{1}(w,u_{1},u_{2},t) -\\ &- j
			\frac{{\partial \boldsymbol{F}_{0}(w,u_{1},u_{2},t)}}{{\partial w}},
			\\
			\frac{{\partial \boldsymbol{F}_{2}(w,u_{1},u_{2},t)}}{{\partial t}} &= \alpha \boldsymbol{F}_{0}(w,u_{1},u_{2},t) + (\boldsymbol{Q} - \boldsymbol{I}\mu_{2})\boldsymbol{F}_{2}(w,u_{1},u_{2},t).
		\end{split}
	\end{equation}   
	Решение системы \eqref{eps_limit_twodim_map} будет получено в форме
	\begin{equation} \label{solution_twodim_map}
		\boldsymbol{F}_{k}(w,u_{1},u_{2},t) = \Phi(w)\boldsymbol{F}_{k}(u_{1},u_{2},t).
	\end{equation}  
	Подставив \eqref{solution_twodim_map} в систему \eqref{eps_limit_twodim_map} и разделив обе части уравнений на $\Phi(w)$, получим
	\begin{equation} \label{preresult_twodim_map}
	\begin{split}
		\frac{{\partial \boldsymbol{F}_{0}(u_{1},u_{2},t)}}{{\partial t}} &= (\boldsymbol{Q}-\boldsymbol{\Lambda}-\alpha\boldsymbol{I})\boldsymbol{F}_{0}(u_{1},u_{2},t) + \mu_{1} e^{ju_{1}}\boldsymbol{F}_{1}(u_{1},u_{2},t)  + \\  &+ \mu_{2}e^{ju_{2}}\boldsymbol{F}_{2}(u_{1},u_{2},t) + j\frac{\Phi'(w) }{\Phi(w)}
		 \boldsymbol{F}_{0}(u_{1},u_{2},t),
		\\
		\frac{{\partial \boldsymbol{F}_{1}(u_{1},u_{2},t)}}{{\partial t}} &= \boldsymbol{\Lambda} \boldsymbol{F}_{0}(u_{1},u_{2},t) +  (\boldsymbol{Q} - \boldsymbol{I}\mu_{1})\boldsymbol{F}_{1}(u_{1},u_{2},t) -\\ &- j\frac{\Phi'(w) }{\Phi(w)}
		 \boldsymbol{F}_{0}(u_{1},u_{2},t),
		\\
		\frac{{\partial \boldsymbol{F}_{2}(u_{1},u_{2},t)}}{{\partial t}} &= \alpha \boldsymbol{F}_{0}(u_{1},u_{2},t) + (\boldsymbol{Q} - \boldsymbol{I}\mu_{2})\boldsymbol{F}_{2}(u_{1},u_{2},t).
	\end{split}
	\end{equation}  
	Исходя из того, что  $\Phi(w)$ имеет смысл асимптотического приближения характеристической функции числа заявок на орбите и имеет вид экспоненты \eqref{Phi_concrete}, система \eqref{preresult_twodim_map} примет следующий вид
	\begin{equation} \label{result_twodim_map}
		\begin{split}
			\frac{{\partial \boldsymbol{F}_{0}(u_{1},u_{2},t)}}{{\partial t}} &= (\boldsymbol{Q}-\boldsymbol{\Lambda}-(\alpha + \kappa)\boldsymbol{I})\boldsymbol{F}_{0}(u_{1},u_{2},t) + \mu_{1} e^{ju_{1}}\boldsymbol{F}_{1}(u_{1},u_{2},t)  + \\  &+ \mu_{2}e^{ju_{2}}\boldsymbol{F}_{2}(u_{1},u_{2},t),
			\\
			\frac{{\partial \boldsymbol{F}_{1}(u_{1},u_{2},t)}}{{\partial t}} &= (\boldsymbol{\Lambda} + \kappa\boldsymbol{I}) \boldsymbol{F}_{0}(u_{1},u_{2},t) +  (\boldsymbol{Q} - \boldsymbol{I}\mu_{1})\boldsymbol{F}_{1}(u_{1},u_{2},t) + \\&+ 0\boldsymbol{F}_{2}(u_{1},u_{2},t),
			\\
			\frac{{\partial \boldsymbol{F}_{2}(u_{1},u_{2},t)}}{{\partial t}} &= \alpha \boldsymbol{F}_{0}(u_{1},u_{2},t) + 0\boldsymbol{F}_{1}(u_{1},u_{2},t) +  (\boldsymbol{Q} - \boldsymbol{I}\mu_{2})\boldsymbol{F}_{2}(u_{1},u_{2},t).
		\end{split}
	\end{equation}  
	Введем следующие обозначения
	\begin{equation*}
		\boldsymbol{FF}(u_{1},u_{2},t) = \{\boldsymbol{F}_{0}(u_{1},u_{2},t),\boldsymbol{F}_{1}(u_{1},u_{2},t),\boldsymbol{F}_{2}(u_{1},u_{2},t)\},
	\end{equation*}  
	\begin{equation*}
		\boldsymbol{G}(u_{1},u_{2})=\begin{bmatrix}
			\boldsymbol{Q}-\boldsymbol{\Lambda}-(\alpha + \kappa)\boldsymbol{I} & \mu_{1}e^{ju_{1}}\boldsymbol{I} &  \mu_{2}e^{ju_{2}}\boldsymbol{I}\\
			\boldsymbol{\Lambda}+\kappa\boldsymbol{I} & \boldsymbol{Q}-\mu_{1}\boldsymbol{I} & 0\\
			\alpha\boldsymbol{I} & 	0 &	\boldsymbol{Q}-\mu_{2}\boldsymbol{I}
		\end{bmatrix}^{T},
	\end{equation*}
	$\boldsymbol{G}(u_{1},u_{2})$ --- транспонированная матрица коэффициентов системы \eqref{result_twodim_map}.
	Тогда получим следующее матричное уравнение
	\begin{equation*}
		\frac{{\partial \boldsymbol{FF}(u_{1},u_{2},t)}}{{\partial t}} =\boldsymbol{FF}(u_{1},u_{2},t)\boldsymbol{G}(u_{1},u_{2}),
	\end{equation*}
	общее решение которого имеет вид
	\begin{equation} \label{diff_twodim_map}
		\boldsymbol{FF}(u_{1},u_{2},t)=\boldsymbol{C}e^{\boldsymbol{G}(u_{1},u_{2})t}.
	\end{equation}
	Для того, чтобы получить единственное решение, соответствующее поведению рассматриваемой системы, примем в рассмотрение начальное условие
	\begin{equation} \label{cauchi_condition_twodim_map}
		\boldsymbol{FF}(u_{1},u_{2},0)=\boldsymbol{R},
	\end{equation}
	где вектор--строка $\boldsymbol{R}$ --- стационарное распределение вероятности состояния прибора, то есть процесса $k(t)$, полученное в теореме \ref{R_theorem}.
	Описав начальное условие, мы можем перейти к решению задачи Коши (\ref{diff_twodim_map}, \ref{cauchi_condition_twodim_map})
	\begin{equation*} 
		\boldsymbol{FF}(u_{1},u_{2},t)=\boldsymbol{R}e^{G(u_{1},u_{2})},
	\end{equation*}
	Поскольку нас интересует распределение вероятностей количества заявок в выходных процессах, необходимо найти маргинальное распределение. Для этого умножаем компоненты вектор--строки $\boldsymbol{FF}(u_{1},u_{2},t)$ на единичный вектор--столбец $\boldsymbol{e}$ размера $N$ и правую часть уравнения на единичный вектор столбец $\boldsymbol{ee}$ размерности $K \cdot N$. Получим
	\begin{equation}\label{approximation_twodim_map}
		\boldsymbol{FF}(u_{1},u_{2},t)\boldsymbol{e}=\boldsymbol{R}e^{\boldsymbol{G}(u_{1},u_{2})t}\boldsymbol{ee}.
	\end{equation}
	Формула \eqref{approximation_twodim_map} является решением рассматриваемой системы. 
\end{proof}

\subsubsection{Переход к явному распределению вероятностей}
Характеристическая функция \eqref{approximation_twodim_map} позволяет нам получит явное распределение вероятностей числа обслуженных заявок процессов $m_{1}(t)$ и $m_{2}(t)$.
Аналогично ранее рассмотренным (\ref{approximation_summary}, \ref{approximation_twodim}), формула \eqref{approximation_twodim_map} содержит матричную экспоненту, к которой мы применяем преобразование подобия \cite{bronson1991matrix}. Процесс разложения матричной экспоненты будет опущен, так как представлен в разделе \ref{distr_find_twodim}.
Вид восстановленного с помощью обратного преобразования Фурье для дискретных величин распределения имеет следующую форму
\begin{equation}\label{distr_map_twodim}
	P(m_{1},m_{2},t) = \dfrac{1}{2\pi}\int_{-\pi}^{\pi}\int_{-\pi}^{\pi} e^{-i \cdot u_{1} \cdot m_{1}} e^{-i \cdot u_{2} \cdot m_{2}}\boldsymbol{FF}(u_{1},u_{2},t)du_{2}du_{2}.
\end{equation}
Полученная формула характеризует вероятность обслуживания $m_{1}$ входящих заявок и $m_{2}$ вызванных заявок к моменту времени $t$ в рассматриваемой системе.
\clearpage