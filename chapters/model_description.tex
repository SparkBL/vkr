\section {Описание модели RQ--системы с обратной связью}

В этом разделе описана модель рассматриваемой системы массового обслуживания, а также полученные аналитические формулы для вычисления характеристик ее работы.

Рассмотрим систему, на вход которой поступает марковский модулируемый пуассоновский поток заявок \cite{meier1987fitting,2019asymptotic,baiocchi1993steady} с диагональной матрицей интенсивностей $\boldsymbol{\Lambda}$ и матрицей инфинитезимальных характеристик $\boldsymbol{Q}$ \cite{yang2013map}. Заявка занимает прибор в случае, если он свободен, тот, в свою очередь, начинает ее обслуживание в течение экспоненциально распределенного времени с параметром $\mu_{1}$. В ситуации, когда прибор уже занят обслуживанием, входящая заявка, после неудавшейся попытки захватить прибор, мгновенно уходит на орбиту и осуществляет там случайную задержу в течение экспоненциально распределенного времени с параметром $\sigma$. В промежутках между обслуживанием входящих заявок прибор самостоятельно их вызывает с интенсивностью $\alpha$ и обслуживает в течение экспоненциально распределенного времени с параметром $\mu_{2}$.
\begin{figure}[H]
	\centering
	\includegraphics[scale=0.5]{system_model.eps}
	\caption{Общий вид RQ--системы с простейшим входящим потоком}
	\label{common_model_fig}
\end{figure}
Введем следующие обозначения: $i(t)$ --- число заявок на орбите в момент времени $t$, $k(t)$ --- состояние прибора: $0$ --- прибор свободен, $1$ --- прибор занят обслуживанием заявки входящего потока, $2$ --- прибор занят обслуживанием вызванной заявки, $m_1(t)$ --- количество обслуженных заявок входящего потока в момент времени $t$, $m_2(t)$ --- количество обслуженных вызванных заявок в момент времени $t$, $n(t)$ --- состояние управляющей цепи MMPP в момент времени $t$.

Для данной модели системы при помощи метода асимптотического анализа при условии $\sigma \xrightarrow{} 0$ было получено асимптотическое приближение характеристической функции числа обслуженных заявок в двух вариантах:
\begin{itemize}
	\item Ситуация, когда процессы $m_1(t)$ и $m_2(t)$ рассматриваются в совокупности:
	\begin{equation}
	\boldsymbol{F}(u,t)\boldsymbol{E}=\boldsymbol{R}e^{\boldsymbol{G}(u)t}\boldsymbol{E}.
	\end{equation}
	\item Ситуация, когда процессы $m_1(t)$ и $m_2(t)$ рассматриваются отдельно друг от друга:
	\begin{equation}
		\boldsymbol{FF}(u_{1},u_{2},t)\boldsymbol{e}=\boldsymbol{R}e^{\boldsymbol{G}(u_{1},u_{2})t}\boldsymbol{ee},
	\end{equation}
\end{itemize}
где $\boldsymbol{G(u)}$ и $\boldsymbol{G(u1,u2)}$ являются транспонированными матрицами системы уравнений Колмогорова
\begin{equation*}
	\boldsymbol{G}(u)=\begin{bmatrix}
		\boldsymbol{Q}-\boldsymbol{\Lambda}-(\alpha + \kappa)\boldsymbol{I} & \mu_{1}e^{ju}\boldsymbol{I} &  \mu_{2}e^{ju}\boldsymbol{I}\\
		\boldsymbol{\Lambda}+\kappa\boldsymbol{I} & \boldsymbol{Q}-\mu_{1}\boldsymbol{I} & 0\\
		\alpha\boldsymbol{I} & 	0 &	\boldsymbol{Q}-\mu_{2}\boldsymbol{I}
	\end{bmatrix}^{T},
\end{equation*}
\begin{equation*}
\boldsymbol{G}(u_{1},u_{2})=\begin{bmatrix}
	\boldsymbol{Q}-\boldsymbol{\Lambda}-(\alpha + \kappa)\boldsymbol{I} & \mu_{1}e^{ju_{1}}\boldsymbol{I} &  \mu_{2}e^{ju_{2}}\boldsymbol{I}\\
	\boldsymbol{\Lambda}+\kappa\boldsymbol{I} & \boldsymbol{Q}-\mu_{1}\boldsymbol{I} & 0\\
	\alpha\boldsymbol{I} & 	0 &	\boldsymbol{Q}-\mu_{2}\boldsymbol{I}
\end{bmatrix}^{T},
\end{equation*}
$\boldsymbol{I}$ --- единичная матрица размерности $N$, вектор--строка $R$ --- стационарное распределение вероятностей процесса $\{k(t),n(t)\}$, где $R_k$ имеет размерность $N$, $\kappa$ --- нормированное среднее число заявок на орбите, $E$ и $ee$ --- единичные вектор--столбцы размерности $N$ и $N \cdot K$ соответственно, где $K$ --- число состояний прибора, а $N$ --- число состояний управляющей цепи MMPP.

На основе полученных асимптотических приближений характеристической функции были получены формулы для вычисления вероятностей обслуживания определенного числа заявок за заданное время при помощи обратного преобразования Фурье. Для этого предварительно была решена задача вычисления матричной экспоненты $e^{\boldsymbol{G}(u1,u2)}$ при помощи преобразования подобия матриц \cite{bronson1991matrix}
\begin{equation*}
	e^{\boldsymbol{G}(u_{1},u_{2})}=\boldsymbol{T}(u_{1},u_{2})\cdot \begin{bmatrix}
		e^{ \Lambda_{1}(u_{1},u_{2})} & 0 &  0\\
		0 & e^{ \Lambda_{2}(u_{1},u_{2})} & 0\\
		0 & 0 &	e^{ \Lambda_{3}(u_{1},u_{2})}
	\end{bmatrix} \cdot \boldsymbol{T}(u_{1},u_{2})^{-1},
\end{equation*}
где $\boldsymbol{T}(u_{1},u_{2})$ --- матрица собственных векторов $\boldsymbol{G}(u_{1},u_{2})$, $\boldsymbol{GJ}(u_{1},u_{2})$ --- диагональная матрица собственных чисел $\Lambda_{n}$ матрицы $\boldsymbol{G}(u_{1},u_{2})$.

В результате были получены следующие формулы:\\
вычисление вероятности обслуживания $m_1$ заявок входящего потока и $m_2$ вызванных заявок к моменту времени $t$
	\begin{equation}\label{distr2}
		P(m_{1},m_{2},t) = \dfrac{1}{(2\pi)^2}\int_{-\pi}^{\pi}\int_{-\pi}^{\pi} e^{-i \cdot u_{1} \cdot m_{1}} e^{-i \cdot u_{2} \cdot m_{2}}\boldsymbol{FF}(u_{1},u_{2},t)\boldsymbol{ee}\hspace{1mm}du_{1}du_{2},
	\end{equation}
вычисление вероятности обслуживания $m$ заявок к моменту времени $t$
\begin{equation}\label{distr}
	P(m,t) = \dfrac{1}{2\pi}\int_{-\pi}^{\pi} e^{-i \cdot u \cdot m} F(u,t)\boldsymbol{E}\hspace{1mm}du.
\end{equation}
Однако вычисление с использованием формул \eqref{distr2} и \eqref{distr} является крайне трудоемкой процедурой, занимающей большое количество времени, поэтому не подходит для проведения численных экспериментов. По этой причине в рамках данной работы предложен оптимизированный способ вычислений вероятностей числа обслуженных заявок рассматриваемой системы с минимальной потерей точности.
\clearpage
