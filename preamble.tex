\usepackage{extsizes}
\usepackage{cmap} % для кодировки шрифтов в pdf
\usepackage[section]{placeins}
\usepackage{fontspec-xetex}
%\usepackage[utf8]{inputenc}
\usepackage[russian]{babel}
\usepackage{slashbox}
%\usepackage{tempora}
\usepackage{graphicx} % для вставки картинок
\graphicspath{ {./figures/} }
\usepackage{amssymb,amsfonts,amsmath,amsthm} % математические дополнения от АМС
\usepackage{indentfirst} % отделять первую строку раздела абзацным отступом тоже
\usepackage[usenames,dvipsnames]{color} % названия цветов
\usepackage{makecell}
\usepackage{csquotes}
\usepackage{pdfpages}
\usepackage{url}
\usepackage{multirow} % улучшенное форматирование таблиц
\usepackage{ulem} % подчеркивания
%\usepackage{titletoc}
\usepackage{tocloft}
\usepackage{tikz}
\usepackage{float}
\usepackage{lastpage}
\usepackage[figure,table]{totalcount}
\usetikzlibrary{chains,shapes.multipart}
\usetikzlibrary{shapes,calc}
\usetikzlibrary{automata,positioning}
\usepackage[left=3cm,right=1.5cm,
top=2cm,bottom=2cm,bindingoffset=0cm]{geometry}
\newtheorem{theorem}{Теорема}


\usepackage{listings}
\usepackage{xcolor}
\lstset { %
breakatwhitespace=false,      
keywordstyle=\color{blue}\ttfamily,
stringstyle=\color{orange}\ttfamily,
showstringspaces=false,
breaklines=true,                
captionpos=b,                    
commentstyle=\color{green}, 
extendedchars=true,              
frame=single,                   
keepspaces=false,                                   
rulecolor=\color{gray},                                          
columns=flexible,
tabsize=2,
literate={0}{{\textcolor{darkpastelgreen}{0}}}{1}%
{1}{{\textcolor{darkpastelgreen}{1}}}{1}%
{2}{{\textcolor{darkpastelgreen}{2}}}{1}%
{3}{{\textcolor{darkpastelgreen}{3}}}{1}%
{4}{{\textcolor{darkpastelgreen}{4}}}{1}%
{5}{{\textcolor{darkpastelgreen}{5}}}{1}%
{6}{{\textcolor{darkpastelgreen}{6}}}{1}%
{7}{{\textcolor{darkpastelgreen}{7}}}{1}%
{8}{{\textcolor{darkpastelgreen}{8}}}{1}%
{9}{{\textcolor{darkpastelgreen}{9}}}{1}%
{.0}{{\textcolor{darkpastelgreen}{.0}}}{2}% 
{.1}{{\textcolor{darkpastelgreen}{.1}}}{2}% 
{.2}{{\textcolor{darkpastelgreen}{.2}}}{2}%
{.3}{{\textcolor{darkpastelgreen}{.3}}}{2}%
{.4}{{\textcolor{darkpastelgreen}{.4}}}{2}%
{.5}{{\textcolor{darkpastelgreen}{.5}}}{2}%
{.6}{{\textcolor{darkpastelgreen}{.6}}}{2}%
{.7}{{\textcolor{darkpastelgreen}{.7}}}{2}%
{.8}{{\textcolor{darkpastelgreen}{.8}}}{2}%
{.9}{{\textcolor{darkpastelgreen}{.9}}}{2}%
}

\newcommand{\specialcell}[2][c]{%
	\begin{tabular}[#1]{@{}c@{}}#2\end{tabular}}

\linespread{1.3} % полуторный интервал
%\renewcommand{\rmdefault}{ftm} % Times New Roman
%\renewcommand{\familydefault}{\rmfamily}
%\fontfamily{ftm}
%\frenchspacing
\setmainfont{Times New Roman}

\usepackage[tableposition=top]{caption}
\usepackage{subcaption}
\DeclareCaptionLabelFormat{gostfigure}{Рисунок #2}
\DeclareCaptionLabelFormat{gosttable}{Таблица #2}
\DeclareCaptionLabelSeparator{gost}{~---~}
\captionsetup{labelsep=gost}
\captionsetup[figure]{labelformat=gostfigure}
\captionsetup[table]{labelformat=gosttable}
\captionsetup[lstlisting]{labelformat=gostfigure}
\renewcommand{\thesubfigure}{\asbuk{subfigure}}
\setlength{\parindent}{1.25cm}


\usepackage[raggedright]{titlesec}


%%Python notebook
\usepackage[most]{tcolorbox}
\definecolor{white}{rgb}{1,1,1}
\definecolor{mygreen}{rgb}{0,0.4,0}
\definecolor{light_gray}{rgb}{0.97,0.97,0.97}
\definecolor{mykey}{rgb}{0.117,0.403,0.713}

\tcbuselibrary{listings}
\newlength\inwd
\setlength\inwd{1.3cm}

\newcounter{ipythcntr}
\renewcommand{\theipythcntr}{\texttt{[\arabic{ipythcntr}]}}

\newtcblisting{pyin}[1][]{%
	sharp corners,
	enlarge left by=\inwd,
	width=\linewidth-\inwd,
	enhanced,
	boxrule=0pt,
	colback=light_gray,
	listing only,
	top=0pt,
	bottom=0pt,
	overlay={
		\node[
		anchor=north east,
		text width=\inwd,
		font=\footnotesize\ttfamily\color{mykey},
		inner ysep=2mm,
		inner xsep=0pt,
		outer sep=0pt
		] 
		at (frame.north west)
		{\refstepcounter{ipythcntr}\label{#1}In \theipythcntr:};
	}
	listing engine=listing,
	listing options={
		aboveskip=1pt,
		belowskip=1pt,
		basicstyle=\footnotesize\ttfamily,
		language=Python,
		keywordstyle=\color{mykey},
		showstringspaces=false,
		stringstyle=\color{mygreen}
	},
}
\newtcblisting{pyprint}{
	sharp corners,
	enlarge left by=\inwd,
	width=\linewidth-\inwd,
	enhanced,
	boxrule=0pt,
	colback=white,
	listing only,
	top=0pt,
	bottom=0pt,
	overlay={
		\node[
		anchor=north east,
		text width=\inwd,
		font=\footnotesize\ttfamily\color{mykey},
		inner ysep=2mm,
		inner xsep=0pt,
		outer sep=0pt
		] 
		at (frame.north west)
		{};
	}
	listing engine=listing,
	listing options={
		aboveskip=1pt,
		belowskip=1pt,
		basicstyle=\footnotesize\ttfamily,
		language=Python,
		keywordstyle=\color{mykey},
		showstringspaces=false,
		stringstyle=\color{mygreen}
	},
}
\newtcblisting{pyout}[1][\theipythcntr]{
	sharp corners,
	enlarge left by=\inwd,
	width=\linewidth-\inwd,
	enhanced,
	boxrule=0pt,
	colback=white,
	listing only,
	top=0pt,
	bottom=0pt,
	overlay={
		\node[
		anchor=north east,
		text width=\inwd,
		font=\footnotesize\ttfamily\color{mykey},
		inner ysep=2mm,
		inner xsep=0pt,
		outer sep=0pt
		] 
		at (frame.north west)
		{\setcounter{ipythcntr}{\value{ipythcntr}}Out#1:};
	}
	listing engine=listing,
	listing options={
		aboveskip=1pt,
		belowskip=1pt,
		basicstyle=\footnotesize\ttfamily,
		language=Python,
		keywordstyle=\color{mykey},
		showstringspaces=false,
		stringstyle=\color{mygreen}
	},
}
%% Python notebook


\titleformat{\section}[block]
{\bfseries\centering}
{\thesection}
{1em}{}

\titleformat{\subsection}[block]
{\bfseries\hspace{\parindent}}
{\thesubsection}
{1em}{}

\titleformat{\subsubsection}[block]
{\bfseries\hspace{\parindent}}
{\thesubsubsection}
{1em}{}


\definecolor{darkpastelgreen}{rgb}{0.01, 0.75, 0.24}

\renewcommand{\cfttoctitlefont}{\normalsize\bfseries\MakeUppercase}
\renewcommand{\cftdotsep}{1}
\renewcommand{\cftsecfont}{\normalfont}
\renewcommand{\cftsecleader}{\cftdotfill{\cftdotsep}}
\renewcommand{\cftsecpagefont}{ \normalfont}
\setlength{\cftbeforesecskip}{0pt}
\setcounter{tocdepth}{3} % задать глубину оглавления — до subsection включительно

\AtBeginDocument{%
\renewcommand{\contentsname}{ОГЛАВЛЕНИЕ}
\renewcommand\refname{\centering СПИСОК ИСПОЛЬЗОВАННОЙ ЛИТЕРАТУРЫ}
	}%
	
%	
%}



\makeatletter
\AtBeginDocument{%
	\let\c@figure\c@lstlisting
	\let\thefigure\thelstlisting
	\let\ftype@lstlisting\ftype@figure % give the floats the same precedence
}
\makeatother

