\usepackage{extsizes}
\usepackage{cmap} % для кодировки шрифтов в pdf
\usepackage[section]{placeins}
\usepackage{fontspec-xetex}
%\usepackage[utf8]{inputenc}
\usepackage[russian]{babel}
\usepackage{slashbox}
%\usepackage{tempora}
\usepackage{graphicx} % для вставки картинок
\graphicspath{ {./figures/} }
\usepackage{amssymb,amsfonts,amsmath,amsthm} % математические дополнения от АМС
\usepackage{indentfirst} % отделять первую строку раздела абзацным отступом тоже
\usepackage[usenames,dvipsnames]{color} % названия цветов
\usepackage{makecell}
\usepackage{csquotes}
\usepackage{pdfpages}
\usepackage{url}
\usepackage{multirow} % улучшенное форматирование таблиц
\usepackage{ulem} % подчеркивания
%\usepackage{titletoc}
\usepackage{tocloft}
\usepackage{tikz}
\usepackage{float}
\usetikzlibrary{chains,shapes.multipart}
\usetikzlibrary{shapes,calc}
\usetikzlibrary{automata,positioning}
\usepackage[left=3cm,right=1.5cm,
top=2cm,bottom=2cm,bindingoffset=0cm]{geometry}
\newtheorem{theorem}{Теорема}


\usepackage{listings}
\usepackage{xcolor}
\lstset { %
 backgroundcolor=\color{black!6},  
basicstyle=\linespread{0.9}\ttfamily,
breakatwhitespace=false,      
breaklines=false,                
captionpos=b,                    
commentstyle=\color{green}, 
extendedchars=true,              
%frame=single,                   
keepspaces=false,             
keywordstyle=\color{blue},      
language=c++,                 
rulecolor=\color{gray},                        
stringstyle=\color{green},                   
title=\lstname 
}


\linespread{1.3} % полуторный интервал
%\renewcommand{\rmdefault}{ftm} % Times New Roman
%\renewcommand{\familydefault}{\rmfamily}
%\fontfamily{ftm}
%\frenchspacing
\setmainfont{Times New Roman}

\usepackage[tableposition=top]{caption}
\usepackage{subcaption}
\DeclareCaptionLabelFormat{gostfigure}{Рисунок #2}
\DeclareCaptionLabelFormat{gosttable}{Таблица #2}
\DeclareCaptionLabelSeparator{gost}{~---~}
\captionsetup{labelsep=gost}
\captionsetup[figure]{labelformat=gostfigure}
\captionsetup[table]{labelformat=gosttable}
\renewcommand{\thesubfigure}{\asbuk{subfigure}}
\setlength{\parindent}{1.25cm}


\usepackage[raggedright]{titlesec}


\titleformat{\section}[block]
{\normalsize\bfseries\centering}
{\thesection}
{1em}{}

\titleformat{\subsection}[block]
{\normalsize\bfseries\hspace{\parindent}}
{\thesubsection}
{1em}{}

\titleformat{\subsubsection}[block]
{\normalsize\bfseries\hspace{\parindent}}
{\thesubsubsection}
{1em}{}



\renewcommand{\cfttoctitlefont}{\normalsize\bfseries\MakeUppercase}
\renewcommand{\cftdotsep}{1}
\renewcommand{\cftsecfont}{\normalfont}
\renewcommand{\cftsecleader}{\cftdotfill{\cftdotsep}}
\renewcommand{\cftsecpagefont}{ \normalfont}
\setlength{\cftbeforesecskip}{0pt}
\setcounter{tocdepth}{3} % задать глубину оглавления — до subsection включительно

\AtBeginDocument{%
\renewcommand{\contentsname}{ОГЛАВЛЕНИЕ}
\renewcommand\refname{\centering СПИСОК ИСПОЛЬЗОВАННОЙ ЛИТЕРАТУРЫ}
	}%
	
%	
%}



\makeatletter
\makeatother

